% !TeX root = thesis.tex
\def\lip#1{\|#1\|_{\text{Lip}}} % Ліпшецева стала

\def\div#1{\text{div}\left(#1\right)} % Оператор дивіргенції

% Функція для малювання повного графа
\newcommand{\completegraph}[1]{%
    \begin{tikzpicture}[scale=3]
        \def\n{#1}
        \def\radius{1}
        % Малюємо вершини по колу
        \foreach \i in {1,...,\n} {
            \pgfmathsetmacro\angle{360/\n * (\i - 1)}
            \node[draw, circle, inner sep=1pt] (v\i) at ({\radius*cos(\angle)}, {\radius*sin(\angle)}) {};
        }
        % З'єднуємо усі вершини
        \foreach \i in {1,...,\n} {
            \foreach \j in {1,...,\n} {
                \ifthenelse{\i < \j}{
                    \draw (v\i) -- (v\j);
                }{}
            }
        }
    \end{tikzpicture}
}