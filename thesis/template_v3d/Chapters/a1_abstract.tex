\abstractUkr
Квалiфiкацiйна робота мiстить: 44 сторiнки, 2 рисунки, 2 таблиці,
9 джерел.

У цій роботі було досліджено зв'язок задачі мінімального потоку (задачі Бекмана) із
задачею оптимального транспортування для випадку дискретного простору (зв'язного невід'ємно зваженого графа).

В ході дослідження було доведено еквівалентність задачі мінімального потоку та задачі оптимального транспортування
на графі. Було запропоновано алгоритм побудови оптимального транспортного плану із оптимального потоку та доведено коректність алгоритму.
Було наведено реалізацію алгоритму мовою Python. Також, було проведено порівняльний аналіз результатів роботи алгоритму з 
іншими методами побудови оптимального транспортного плану.

\MakeUppercase{ГРАФ, ПОТІК НА ГРАФІ, ОПТИМАЛЬНИЙ ТРАНСПОРТНИЙ ПЛАН}

\abstractEng
The qualification work contains: 44 pages, 2 figures, 2 tables, and 9 citations.

This work investigates the relationship between the minimum flow problem (the Beckmann problem) and 
the optimal transport problem in the case of a discrete space (a connected non-negatively weighted graph).

In the course of the research, the equivalence of the minimum flow problem and the optimal transport problem on a graph was proven. 
An algorithm for constructing an optimal transport plan from an optimal flow was proposed,
 and the correctness of the algorithm was proven.
An implementation of the algorithm in Python was presented. A comparative analysis of the algorithm's results with other
methods of constructing the optimal transport plan was also conducted.

\MakeUppercase{GRAPH, FLOW ON A GRAPH, OPTIMAL TRANSPORTATION PLAN}
\clearpage