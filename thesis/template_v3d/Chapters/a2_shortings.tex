%!TEX root = ../thesis.tex
% ===================== Запхнути це в оточення array ==========================
\begin{tblr}{}
    &$\mathbb{R}_+$  &--- &множина невід'ємних дійсних чисел. \\

    &$\delta_A$      &--- &міра Дірака множини $A$. \\

    &$C_{uv}$        &--- &шлях мінімальної ваги між вершинами $u, v$. \\

    &$\mathbf{n}$    &--- &вектор зовнішньої нормалі. \\

    &$\partial A$    &--- &межа множини $A$. \\

    &$d H^n$         &--- &$n$-вимірна міра Хаусдорфа. \\

    &$d\lambda^n$    &--- &$n$-вимірна міра Лебега. \\

    &$\mu[A]$        &--- &значення міри $\mu$ на множині $A$. \\

    &$\Pi(\mu, \nu)$ &--- &множина каплінгів мір $\mu$ та $\nu$. \\

    &$\otimes$       &--- &добуток Кронекера. \\

    &$L^1(\mu)$      &--- &множина абсолютно інтегровних функцій відносно міри $\mu$. \\

    &$\lip{\varphi}$ &--- &скорочене позначення сталої Ліпшиця $\sup_{x \neq y} \dfrac{|\varphi(x) - \varphi(y)|}{d(x, y)}$. \\

    &$\varphi^c$     &--- &c-перетворення функції $\varphi: X \to \mathbb{R}$ визначене як \\
    &&&$\varphi^c(y) = \inf_{x \in X}[c(x, y) - \varphi(x)]$. \\

    &$\varphi^{cc}$  &--- &cc-перетворення функції $\varphi: X \to \mathbb{R}$ визначене як \\
    &&&$\varphi^{cc}(x) = \inf_{y \in Y}[c(x, y) - \varphi^c(y)]$.
\end{tblr}