%!TEX root = ../thesis.tex

\textbf{Актуальність дослідження.} Актуальність даного дослідження зумовлена широким використанням графа для моделювання
зв'язків між містами і побудови транспортних маршрутів в логістиці. В загальному випадку побудова оптимальних транспортних
маршрутів шляхом розв'язання лінійних задач не є оптимальною за часом. Мотивацією для запропонованого в цій роботі підходу
є те, що побудова оптимального потоку є доволі швидкою процедурою і вимагає менше пам'яті.

\textbf{Метою дослідження} є побудова і реалізація алгоритму перетворення потоку на графі на план транспортування. Для
досягнення мети необхідно вирішити такі завдання:

\begin{enumerate}
    \item провести огляд опублікованих джерел за тематикою дослідження;
    \item довести еквівалентність поставлених задач;
    \item отримати процедуру перетворення оптимального потоку на оптимальний транспортний план;
    \item перевірити одержані результати експериментально.
\end{enumerate}

\textit{Об'єктом дослідження} є логістично-економічні процеси.

\textit{Предметом дослідження} є математичні методи, графові моделі і алгоритми дослідження логістично-економічних процесів.

При розв'язанні поставлених задач використовувались такі \textit{методи дослідження}: методи дискретної математики (теорія графів),
функціонального аналізу (теорії міри) та комп'ютерного моделювання (проведення обчислювальних експериментів).

\textbf{Практичне значення} полягає в прискоренні побудови оптимального транспортного плану на графі.