%!TEX root = ../thesis.tex
\chapter{Задача оптимального транспортування та задача пошуку оптимального потоку}
\label{chap:review}
В першому розділі розглянуто поняття задачі оптимального транспортування та 
поняття задачі оптимального потоку для неперервного випадку.
Розглянуто два підходи до формулювання задачі оптимального транспортування, і зв'язок між ними.

\section{Задача оптимального транспортування}
\label{sec:optimal-transport}
Сформулюємо основні поняття теорії оптимального транспортування.

\begin{definition}[Образ міри при відображенні]
    \label{def:push-forward}
    Нехай задана міра $\mu$ на вимірному просторі $X$ і вимірне відображення $T : X \to Y$.
    \textbf{Образом міри $\mu$ при відображені $T$} називається міра $\nu$ на $Y$, для якої справедливо:
    $$
        \forall A \subset Y, A\text{ -- вимірна}: \nu[A] = \mu\left[T^{-1}(A)\right].
    $$
    Позначають: $\nu = T \# \mu$.
\end{definition}


\begin{definition}
    \label{def:coupling}
    Нехай дано міру $\mu$ на множині $X$ і міру $\nu$ на множині $Y$. \textbf{Каплінгом} мір $\mu$ та $\nu$ називається
    міра $\pi$ на $X \times Y$ така, що для довільних вимірних множин $A \subset X$ та $B \subset Y$ виконується
    \begin{eqnarray}
        \pi[A \times Y] = \mu[A]; \\
        \pi[X \times B] = \nu[B].
    \end{eqnarray}
\end{definition}

Множину усіх каплінгів мір $\mu$ та $\nu$ будемо позначати $\Pi(\mu, \nu)$.

\subsection{Задача Монжа}
\label{sec:monge-problem}
Гаспар Монж у своїй роботі 1781 року~\cite{monge1781} сформулював задачу оптимального перенесення купи ґрунту $X$ 
в яму $Y$ того ж об'єму. Перенесення визначається за допомогою деякої функції $T : X \to Y$, що не формально
позначає в яку точку ями $y \in Y$ треба перевезти ґрунт з точки $x \in X$.

Формулювання задачі Монжа має наступний вигляд:

\begin{definition}[Задача Монжа]
    \label{def:monge-problem}
    Дано розподіли $\mu$ на $X$ та $\nu$ на $Y$ і деяка функція ціни $c : X \times Y \to \mathbb{R}_+ \cup \{+\infty\}$,
    яка визначає вартість перевезення ґрунту з точки $x \in X$ в точку $y \in Y$. 
    Необхідно знайти функцію $T^* : X \to Y$ таку, що $T^* \# \mu = \nu$ і функціонал
    $$
        \int_X c\left(x, T^*(x)\right)\, d\mu(x),
    $$
    досягає мінімуму.
\end{definition}

Основною проблемою такого формулювання є той факт, що маса деякої точки $x \in X $ не може бути розділена між двома місцями
в ямі $y_1, y_2 \in Y$.

\subsection{Задача Канторовича}
\label{sec:kantorovich-problem}

У 1942 Леонід Канторович сформулював задачу транспортування товару до споживачів~\cite{kantorovich2006}.

У формалізації Канторовича ми маємо дві множини $X$ та $Y$, що відповідно є множинами <<товару>> та <<споживачів>>;
щільності <<виробництва товару>> і <<споживання>> відповідно рівні $\mu$ та $\nu$.
Задача полягає в пошуку оптимального планe перевезення $\pi \in \Pi(\mu, \nu)$ усього <<товару>> до <<споживачів>>, за умови, що
ціна перевезення одиниці товару з точки $x \in X$ до споживача у точці $y \in Y$ рівна $c(x, y)$.

\begin{definition}[Задача Канторовича]
    \label{def:kantorovich-problem}
    Дано розподіли $\mu$ на $X$ та $\nu$ на $Y$ і деяка функція ціни $c : X \times Y \to \mathbb{R}_+ \cup \{+\infty\}$. 
    Необхідно знайти міру $\pi \in \Pi(\mu, \nu)$, яка мінімізує функціонал
    $$
        \int_{X \times Y} c(x, y)\, d\pi.
    $$
\end{definition}

На сьогоднішній день формулювання вище має назву \textit{задачі Монжа-Канторовича}.

\subsection{Дуальність}
Задача Канторовича має дуальне формулювання.

Визначимо множину:
$$
    \Phi_c = \{(\varphi, \psi) \in L^1(\mu) \times L^1(\nu) : \forall (x, y) \in X \times Y : \varphi(x) + \psi(y) \le c(x, y)\},
$$
і визначимо функціонал $J(\varphi, \psi)$ як:
$$
    J(\varphi, \psi) = \int_X \varphi\, d\mu + \int_Y \psi\, d\nu.
$$
Тоді справедлива наступна теорема.

\begin{theorem}[Дуальність Канторовича~\cite{villani2003}]
    \label{theorem:kantorovich-duality}
    Нехай $X$ та $Y$ --- дві компактні множини з мірами $\mu$ та $\nu$, відповідно. 
    Нехай $c : X \times Y \to \mathbb{R}_+ \cup \{+\infty\}$ --- напівнеперервна знизу функція ціни. Тоді:
    $$
        \inf_{\pi \in \Pi(\mu, \nu)} \int_{X \times Y} c(x, y)\, d\pi(x, y) = \sup_{(\varphi, \psi) \in \Phi_c} J(\varphi, \psi).
    $$
\end{theorem}

Ця теорема є фундаментом для подальшого дослідження як методів розв'язку задачі Канторовича,
так і зв'язку з іншими задачами оптимізації. 

Зокрема, важливим наслідком теореми~\ref{theorem:kantorovich-duality} є наступна теорема.

\begin{theorem}[Дуальність Канторовича-Рубінштейна~\cite{villani2003}]
    \label{theorem:kr-duality}
    Нехай $(X, \rho)$ --- компактний метричний простір і задані дві міри $\mu$ та $\nu$ на $X$.
    Для функції ціни $c(x, y) = \rho(x, y)$ виконується:
    \begin{equation}
        \begin{aligned}
            \label{eq:kr-duality}
            \inf_{\pi \in \Pi(\mu, \nu)} \int_{X \times Y} \rho(x, y)\, d\pi(x, y) = \\
            = \sup_{\varphi \in L^1(\mu - \nu)} 
            \left\{
                \int_X \varphi\, d(\mu - \nu) : \lip{\varphi} \le 1
            \right\}.
        \end{aligned}
    \end{equation}
\end{theorem}

\section{Задача пошуку мінімального потоку}
\label{sec:min-flow-problem}

Нехай дано деяку достатньо гладенька і зв'язну область $\Omega \subset \mathbb{R}^2$,
яку можна інтерпретувати як деяке <<місто>>.
На множині $\Omega$ визначені дві ймовірнісні міри $\mu$, $\nu$ --- відповідно міра локального попиту 
та локальної пропозиції.
В цих позначеннях можна вважати, що міра $(\mu - \nu)$ є локальною мірою надлишкового попиту.

Будемо вважати, що трафік товару моделюється як деяке векторне поле $\mathbf{Y} : \Omega \to \mathbb{R}^2$. 
Таким чином напрям $\mathbf Y(x)$ визначає напрям руху товару в точці $x \in \Omega$,
а $\left|\mathbf{Y}(x)\right|$ визначає інтенсивність цього руху.

Природньо вважати, що виконується аналог закону збереження маси:
кількісно витік споживачів із довільної області $K \subset \Omega$ рівний надлишковому попиту в цій області:
\begin{equation*}
    \int_{\partial K} \mathbf{Y} \cdot \mathbf{n}\, dH = (\mu - \nu)(K).
\end{equation*} 
Локально це еквівалентно рівності в слабкому сенсі:
\begin{equation}
    \label{eq:equilibrium}
    \div{\mathbf{Y}} = (\mu - \nu),
\end{equation}
тобто
$$
    \int_{\partial K} \mathbf{Y} \cdot \mathbf{n}\, dH = \iint_{K} \div{Y} d\lambda^2 =
    \iint_{K} d(\mu - \nu)
$$

Також будемо вважати наше місто ізольованим:
\begin{equation}
    \label{eq:isolation}
    \mathbf{Y} \cdot \mathbf{n} = 0\text{ на }\partial \Omega.
\end{equation}

\begin{definition}[Задача пошуку мінімального потоку~\cite{cocv, beckmann52}]
    \label{def:min-flow}
    Нехай дано дві міри $\mu$ та $\nu$ на множині $\Omega$
    і деяка неспадна функція $g : \mathbb{R}_+ \to \mathbb{R}_+ \cup \{+\infty\}$.
    Необхідно знайти таке векторне поле $\mathbf{Y}^*$, що задовольняє умовам~(\ref{eq:equilibrium})--(\ref{eq:isolation})
    і мінімізує функціонал:
    \begin{equation}
        \label{eq:min-flow}
         \int_\Omega g\left(\left|\mathbf{Y}(\omega)\right|\right)\, d\lambda^2.
    \end{equation}
       
    Векторне поле $\mathbf{Y}^*$, на якому досягається мінімум~(\ref{eq:min-flow}), називають \textbf{мінімальним потоком}.
\end{definition}

В цій дещо спрощеній моделі ми вважаємо, що ціна транспортування одиниці товару не залежить від інтенсивності (нема заторів).
Також у якості функції $g$ оберемо функцію $g(t) = t$ для $t \in \mathbb{R}_+$; в багатьох випадках це доволі природній вибір.

Тоді задача пошуку мінімального потоку приймає вигляд
$$
    \int_{\Omega} \left|\mathbf{Y}(x)\right|\, d\lambda^2 \to \min.
$$

Неформально ми шукаємо такий потік, сумарна інтенсивність якого є мінімальною.

\section{Зв'язок задачі Канторовича з пошуком мінімального потоку}
Задача пошуку мінімального потоку~\ref{def:min-flow} полягає у
знаходженні шляхів, якими треба транспортувати товар.
З іншого боку, задача Канторовича~\ref{def:kantorovich-problem} полягає в
знаходженні способів перевезення одиниці товару до споживача.

Природним чином постає питання про зв'язок цих задач.

\begin{theorem}[Еквівалентність задачі пошуку мінімального потоку і задачі Монжа-Канторовича~\cite{otam}]
    \label{theorem:min-flow-kantorovich}
    Нехай $\mu, \nu$ -- дві ймовірнісні міри на $\Omega$, $\rho(x, y) = |x - y|_2$.
    Тоді:
    $$
        \inf_{\pi \in \Pi(\mu, \nu)} \int_{\Omega^2} \rho(x, y)\, d\pi(x, y) = \inf
        \left\{
            \int_\Omega \left|\mathbf{Y}(x)\right|\, d\lambda^2 
            : \mathbf{Y} \text{ задовольняє } (\ref{eq:equilibrium})-(\ref{eq:isolation})
        \right\}.
    $$
\end{theorem}

Отже, постає питання про те, яким чином ми можемо отримати оптимальний план транспортування
для задачі Канторовича~\ref{def:kantorovich-problem}, використовуючи деякий мінімальний потік, і навпаки.

\chapconclude{\ref{chap:review}}

В цьому розділу було розглянуто основні означення теорії оптимального транспортування і визначення задачі мінімального потоку.

Було розглянуто випадок, коли ціна в задачі оптимального транспортування задається евклідовою метрикою. 
В цьому випадку можна побудувати еквівалентність між задачами оптимального транспортування та мінімального потоку.

Наступний розділ буде присвячено дослідженню еквівалентності цих задач у випадку графа і метрики, індукованої вагами цього графа.