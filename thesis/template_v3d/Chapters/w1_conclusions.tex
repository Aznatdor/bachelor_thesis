%!TEX root = ../thesis.tex
У цій роботі було дослідженно математичну модель задачі побудови оптимального плану транспортування і
задачі побудови мінімального потоку.

У параграфі~\ref{sec:optimal-transport} було розглянуто неперевне формулювання 
задачі Монжа-Канторовича~(\ref{def:kantorovich-problem}) тоді як параграф~\ref{sec:min-flow-problem}
було присвячено задачі пошуку мінімального потоку.

У розділі~\ref{chap:graph} було розглянуто дискретні формулювання цих задач. В якості природнього дискретного аналогу
метричного простору було обрано зв'язний не орієнтований граф з невід'ємними вагами. Було розроблено аналоги
диференціальних операторів для графу, а наведені аналоги були використані при формулюванні задачі пошуку мінімального потоку.

Також було доведено еквівалентність поставлених задач на графі у теоремі~\ref{theorem:equiv}.

Розділ~\ref{chap:experiment} було присвяченно питанню чисельної перевірки одержаних теоретичних результатів. Було
наведено твердження, що дозволило розв'язати задачу Монжа-Канторовича на графі як задачу
лінійного програмування і посилання на алгоритм, для розв'язання задачі мінімального потоку.

Процедура з доведення теореми~\ref{theorem:equiv} була запрограмована мовою Python і наведена в додатку.
Коректність роботи процедури була перевірена на двох прикладах. 
Проведені чисельні експерименти показали, що побудовані двома способами плани є оптимальними, хоча можуть відрізнятись.