%!TEX root = ../thesis.tex
\chapter{Порівняння результатів}
\label{chap:experiment}

У цьому розділі наведено короткі теоретичні відмості про алгоритмічні розв'язки задачі оптимального траснспортування та 
мінімального потоку. Наведено порівняння резлутатів знаходження оптимального транспортування на пряму і використовуючи мінімальний
потік. Наведено аналіз отриманих результатів.

\section{Алгоритмічне підґрунтя}
Коротко розглянемо теоретичні відомості, що дозволять чисельно порівняти два підходи отримання оптимального транспортного плану:
розв'язок лінійной задачі і перетворення мінімального потоку

\subsection{Задача Канторовича}
Наведемо спосіб перетворити дискретну задачу Канторовича~(\ref{def:kantorovich-problem-d}) у задачу лінійного програмування У
стандартній формі.

\begin{claim}[Задача Канторовича як задача лінійного програмування~\cite{cot}]
    Нехай $\mathbf{C}$ -- матриця цін, $\mathbf{a}, \mathbf{b}$ -- дві дискретні міри
    та $\mathbf{I}_n$ -- одинична матриця $n \times n$. Визначимо матрицю
    $$
    \mathbf{A} = 
    \begin{bmatrix}
    \mathbb{1}_n^\top \otimes \mathbf{I}_m \\
    \mathbf{I}_n \otimes \mathbb{1}_m^\top
    \end{bmatrix}.
    $$
    Визначимо вектор $\mathbf{c}$, як вектор, отриманий шляхом з'єднання колонок матриці $\mathbf{C}$ у вектор-стовпчик
    та вектор $\mathbf{k}$ як
    $$ \mathbf{k} =
        \begin{bmatrix}
            \mathbf{a} \\
            \mathbf{b}
        \end{bmatrix}.
    $$
    Тоді
    \begin{equation}
        \label{eq:kantorovich-linear}
        \min_{\mathbf{P} \in U\left(\mathbf{a}, \mathbf{b}\right)} \sum_{i,j} \mathbf{P}_{ij} \cdot \mathbf{C}_{ij} =
        \min_{\substack{
            \mathbf{p} \in \mathbb{R}_+^{nm} \\
            \mathbf{A}\mathbf{p} = \mathbf{k}
        }}
        \mathbf{c}^\top \mathbf{p}.
    \end{equation}
\end{claim}

Відповідно пошук оптимального плану транспортування можна звести до розв'язання задачі лінійного
програмування~(\ref{eq:kantorovich-linear})

\subsection{Задача пошуку мінімального потоку}

Визначимо \textbf{залишковий граф} (\textit{англ.} Residual Network).
\begin{definition}[Залишковий граф]
    Для графа $G = (V, E, w)$ і потоку $\mathbf{s}$ визначимо \textbf{залишковий граф} $G^s = (V, E^s, w^s)$, де
    $E^s = \{e \in E : \mathbf{s}_e > 0\}$ і 
    $$
        w^s_{u, v} = \begin{cases}
            w_{u, v}, &\text{якщо }(u, v) \in E \\
            -w_{v, u}, &\text{якщо }(v, u) \in E
        \end{cases}.
    $$
\end{definition}

Наступне твердження є фундаментом алгоритму усунення циклів
(\textit{англ.} Cycle Cancelling algorithm)~\cite{cycle}, який буде використаний в якості
алгоритму пошуку мінімального потоку на графі.

\begin{theorem}[Критерій мінімальності потоку~\cite{busacker1965finite}]
    Потік $\mathbf{s}$ на графі $G$ мінімальним тоді і лише тоді, коли граф $G^s$ не має циклів від'ємної ціни.
\end{theorem}

Отримавши оптимальний потік, ми можемо отримати оптимальний план, використавши процедуру,
наведену в доведенні теореми~\ref{theorem:equiv}.

\section{Порівняння результатів}
Для порівняння роботи алгоритмів розглянемо декілька графів із <<складною>> системою ребер. Гарним прикладом таких графів
є так звані \textbf{повні графи}, що мають усі можливі ребра. Ваги ребер, розподіли $\mathbf{a}$ та $\mathbf{b}$
будуть зазначені окремо в таблицях, щоб не засмічувати рисунки. Далі через $K_n$ будемо позначати повний граф на
$n$ вершинах

Спочатку розглянемо повний граф $K_5$ на 5 вершинах
\begin{figure}[h]
    \centering
    \completegraph{5}
    \caption{Повний граф на 5 вершинах}
\end{figure}
з відповідними розподілами $\mathbf{a}, \mathbf{b}$ (таблиця~\ref{tab:ex1-ds}) 
\begin{table}[h]
    % Шість колонок три ряди
    \centering
    \begin{tblr}{
        colspec={XXXXXX},
        hlines={-}{},
        vlines={-}{},
        row{1}={c, m, font=\bfseries},
        rows={c, m}
        }
        $v$         &   1&  2&   3&  4&  5 \\
        $\mathbf{a}$   &   6&  4&  10&  8&  9 \\
        $\mathbf{b}$   &   4&  2&  14&  9&  8 \\
    \end{tblr}
    \caption{Таблиця значень розподілів $\mathbf{a}, \mathbf{b}$ на вершинах графа $K_5$}
    \label{tab:ex1-ds}
\end{table}
та вагами
$$
\mathbf{C} = 
\begin{pmatrix}
     0&  7&   6& 10&  5 \\ 
     7&  0&   7&  3&  8 \\
     8&  7&   0&  5&  2 \\
    10&  3&   5&  0&  9 \\
     5&  8&   2&  9&  0
\end{pmatrix}.
$$
Розв'язавши лінійну задачу~(\ref{eq:kantorovich-linear}), ми отримаємо наступний оптимальний план
$$
\mathbf{P}_T^* = 
\begin{pmatrix}
    0& 0& 2& 0& 0 \\
    0& 0& 1& 1& 0 \\
    0& 0& 0& 0& 0 \\
    0& 0& 0& 0& 0 \\
    0& 0& 1& 0& 0 
\end{pmatrix},
$$
з ціною транспортування $24$. 

Порівняємо результат, застосувавши алгоритм усунення циклів і перетворивши оптимальний потік 
на транспортний план. Отримаємо
$$
\mathbf{P}_F^* =
\begin{pmatrix}
    0& 0& 2& 0& 0 \\
    0& 0& 1& 1& 0 \\
    0& 0& 0& 0& 0 \\
    0& 0& 0& 0& 0 \\
    0& 0& 1& 0& 0 
\end{pmatrix},
$$
з ціною $24$.

Таким чином обидва алгоритми отримали плани, ціна яких рівна $24$.

Проведемо порівняння підходів на більшому графі $K_{7}$. 
\begin{figure}[h]
    \centering
    \completegraph{7}
    \caption{Повний граф на 7 вершинах}
\end{figure}
Розподіли $\mathbf{a}, \mathbf{b}$ подані в таблиці~\ref{tab:ex2-ds}.
Матрицю цін, розподіли $\mathbf{a}, \mathbf{b}$
зробимо менш випадковими. Це дозволить алгоритмам знайти різні оптимальні розв'язки.
$$
\mathbf{C} =
\begin{pmatrix}
    0&2&1&2&1&2&1 \\
    2&0&2&1&2&1&2 \\
    1&2&0&2&1&2&1 \\
    2&1&2&0&2&1&2 \\
    1&2&1&2&0&2&1 \\
    2&1&2&1&2&0&2 \\
    1&2&1&2&1&2&0
\end{pmatrix}.
$$

\begin{table}[h]
    % Вісім колонок три ряди
    \centering
    \begin{tblr}{
        colspec={XXXXXXXX},
        hlines={-}{},
        vlines={-}{},
        row{1}={c, m, font=\bfseries},
        rows={c, m}
        }
        $v$         &    1&   2&   3&   4&   5&   6&  7\\
        $\mathbf{a}$   &    1&   0&   1&   1&   0&   1&  0\\
        $\mathbf{b}$   &    0&   1&   0&   1&   1&   0&  1
    \end{tblr}
    \caption{Таблиця значень розподілів $\mathbf{a}, \mathbf{b}$ на вершинах графу $K_{7}$}
    \label{tab:ex2-ds}
\end{table}

Розв'язання лінійної задачі дає такий оптимальний транспортний план

$$
\mathbf{P}^*_T =
\begin{pmatrix}
    0&  0& 0& 0& 1& 0& 0\\
    0&  0& 0& 0& 0& 0& 0\\
    0&  0& 0& 0& 0& 0& 1\\
    0&  0& 0& 0& 0& 0& 0\\
    0&  0& 0& 0& 0& 0& 0\\
    0&  1& 0& 0& 0& 0& 0\\
    0&  0& 0& 0& 0& 0& 0
\end{pmatrix}.
$$
Ціна транспортування вище рівна $3$. В свою чергу, за допомогою алгоритму усунення циклів було побудовано план

$$
\mathbf{P}^*_F =
\begin{pmatrix}
    0&  0& 0& 0& 0& 0& 1\\
    0&  0& 0& 0& 0& 0& 0\\
    0&  0& 0& 0& 1& 0& 0\\
    0&  0& 0& 0& 0& 0& 0\\
    0&  0& 0& 0& 0& 0& 0\\
    0&  1& 0& 0& 0& 0& 0\\
    0&  0& 0& 0& 0& 0& 0
\end{pmatrix}.
$$
Ціна транспортування також $3$. Хоча план транспортування відрізняється, легко перевірити, що він дійсно переносить
$\mathbf{a}$ на $\mathbf{b}$.

\chapconclude{\ref{chap:experiment}}
В цьому розділі було розглянуто алгоритмічне підґрунтя для розв'язання задач оптимального транспортування і 
мінімального потоку.

Було отримано результати, що свідчать про коректність процедури. 
В першому випадку плани траснспортування співпадають, але в другому ми наочно бачимо різницю. Тим не менш, як було доведено і
перевірено, оптимальна ціна транспортування лишалась однаковою.