%!TEX root = ../thesis.tex
\chapter{Оптимальне транспортування і мінімальний потік на графі}
\label{chap:graph}

В цьому розділі будуть сформульовані аналоги задачі Канторовича~(\ref{def:kantorovich-problem})
та задачі пошуку мінімального потоку~(\ref{def:min-flow}) для графу.

Буде доведено еквівалентність цих задач. Також, буде одержано процедуру побудови
оптимального плану транспортування з мінімального потоку.

\section{Задача Канторовича на скінченній множині}
Спочатку наведемо визначення каплінгу дискретних мір. В межах цієї роботи будемо вважати,
що міра на множині з $n < \infty$ елементів задається вектором з невід'ємними компонентами.

\begin{definition}[Каплінг дискретних мір]
    Нехай дано дві дискретні міри $\mathbf a \in \mathbb{R}_+^n$ та $\mathbf b \in \mathbb{R}_+^m$. Також покладемо
    $\mathbb{1}_n$ -- вектор-стовпчик одиниць.

    \textbf{Каплінгом} двох дискретних мір називається матриця $\mathbf{P} \in \mathbb{R}_{+}^{n \times m}$ така, що
    \begin{eqnarray*}
        \mathbf{P}\mathbb{1}_m = \mathbf a; \\
        \mathbf{P}^\top \mathbb{1}_n = \mathbf b
    \end{eqnarray*}
\end{definition}

Множину усіх каплінгів двох дискретних мір $\mathbf a, \mathbf b$ будемо позначати $\mathbf{U}\left(\mathbf a, \mathbf b\right)$

Тепер наведемо означення задачі Канторовича на випадок двох скінченних множин.

\begin{definition}[Лінійна програма Канторовича~\cite{cot}]
    \label{def:kantorovich-problem-d}
    Нехай дано дві скінченні множини $X$ та $Y$, причому $|X| = n, |Y| = m$. Також нехай дано дві дискретні міри
    $\mathbf a \in \mathbb{R}_+^n$ та $\mathbf b \in \mathbb{R}_+^m$ і матриця цін $\mathbf{C} \in \mathbb{R}_+^{n \times m}$.

    Необхідно зайти матрицю $\mathbf{P} \in \mathbf{U}\left(\mathbf a, \mathbf b\right)$ таку, що функціонал
    $$
        \sum_{i, j} \mathbf{P}_{ij} \cdot \mathbf{C}_{ij},
    $$
    досягає мінімуму.
\end{definition}

Означення вище є частковим випадком~(\ref{def:kantorovich-problem}), коли обидві міри $\mu$ та $\nu$ є
сумами мір Дірака.

Таким чином для означення вище також справедливі аналоги дуальних теорем~(\ref{theorem:kantorovich-duality}--\ref{theorem:kr-duality}).

\section{Задача пошуку мінімального потоку на графі}

Надалі ми будемо працювати із зваженими графами $G = (V, E, w)$, де $V$ -- скінченна множина вершин графу,
$E \subset V \times V$ -- множина неорієнтованих ребер графу, тобто $(u, v) \in E \iff (v, u) \in E$, 
$w : E \to \mathbb{R}_+ \cup \{+\infty\}$ -- функція ваги ребра;
без втрати загальності вважаємо вершини графу $V$ пронумерованими від $1$ до $n := |V|$.

Визначимо потік на графі.
\begin{definition}[Потік на графі]
    \textbf{Потоком на графі} $G = (V, E, w)$ будемо називати довільне відображення $s : E \to \mathbb{R}_+$.
\end{definition}

Без втрати загальності можемо вважати, що довільна функція $f : V \to \mathbb{R}$ 
є деяким вектором $\mathbf{f} \in \mathbb{R}^{|V|}$, для якого виконується $\forall v \in V: f(v) = \mathbf{f}_v$.

Навідміну від непервоного випадку, в якому $\mathbf{f}$ є вектоним полем, потік на графі має тільки <<інтенсивність>>; напрям руху
визначається ребром, що є аргументом потоку.

\subsection{Диференціальні оператори на графі}
Сформулюємо аналоги диференціальних операторів для графу. Такий підхід дозволить природним шляхом узагальнити поняття
задачі мінімального потоку на випадок графу.

\begin{definition}[Оператор градієнту]
    \label{def:grad}
    Для довільної функції вершини графу $\mathbf f \in \mathbb{R}^{|V|}$ визначимо оператор градієнту
    $\nabla : \mathbb{R}^{|V|} \to \mathbb{R}^{|E|}$ як:
    $$
        \forall (i, j) \in E: (\nabla \mathbf f)_{ij} := \mathbf{f}_i - \mathbf{f}_j.
    $$
\end{definition}

\begin{definition}[Оператор дивергенції]
    \label{def:div}
    Для довільного потоку $\mathbf s \in \mathbb{R}^{|E|}$ визначимо оператор дивергенції
    $\text{div} : \mathbb{R}^{|E|} \to \mathbb{R}^{|V|}$ як:
    $$
        \forall i \in V : \div{s}_i := \sum_{j: (i, j) \in E} \mathbf{s}_{ij} - \mathbf{s}_{ji}.
    $$
\end{definition}

Наступне твердження доводить зв'язок між визначеннями вище.

\begin{claim}[Коректність визначення]
    Нехай дан граф $G = (V, E, w)$ і дві фунції $f : V \to \mathbb{R}$ та $g : E \to \mathbb{R}$.
    Визначимо скалярне множення $\langle f, g \rangle := \sum_{v \in V} f_v \cdot g_v$.

    Тоді справедлива рівність:
    \begin{equation*}
        \langle \nabla f, g \rangle = \langle f, \textnormal{div}(g) \rangle.
    \end{equation*}
\end{claim}
\begin{proof}
    Розкриємо скалярний добуток $\langle \nabla f, g \rangle$:
    \begin{align*}
        &\langle \nabla f, g \rangle = \sum_{e \in E} (\nabla f)_e \cdot g_e =
        \sum_{v \in V} \sum_{u : (v, u) \in E} (\nabla f)_{uv} \cdot g_{uv} = \\
        &=\sum_{v \in V} \sum_{u : (v, u) \in E} (f_u - f_v) \cdot g_{uv} =
        \sum_{v \in V} \sum_{u : (v, u) \in E} f_u \cdot g_{uv} - \\
        &-\sum_{v \in V} \sum_{u : (v, u) \in E} f_v \cdot g_{uv}.
    \end{align*}
    Тепер змінимо порядок підсумовування першої суми. Отримаємо:
    \begin{align*}
        &\sum_{v \in V} \sum_{u : (v, u) \in E} f_v \cdot g_{vu} - \sum_{v \in V} \sum_{u : (v, u) \in E} f_v \cdot g_{uv} = \\
        &= \sum_{v \in V} f_v \sum_{u : (v, u) \in E} (g_{vu} - g_{uv}) =
        \sum_{v \in V} f_v \cdot \text{div}(g) = \langle f, \text{div}(g) \rangle
    \end{align*}
\end{proof}

Таким чином, наведені вище означення мають поведінку, 
притаману неперервним операторам градієнту і дивергенції: оператор дивергенції є спряженим оператор до оператору градієнту
відносно рівномірного розподілу на графі.

\subsection{Потік на графі}
Розглянемо зв'язний граф $G = (V, E, w)$, який моделює деяке місто.
Природньо вважати, що ваги ребер моделюють відстань між районами міста, зокрема $\forall e \in E: w(e) \ge 0$.

На множині вершин визначимо дискретні міри $\mathbf a$ та $\mathbf b$, що, аналогічно до секції~\ref{sec:min-flow-problem}, є
відповідно мірами локального попиту та пропозиції.

Визначимо аналог закону збереження маси~(\ref{eq:equilibrium}): для довільної вершини $v \in V$:
\begin{equation}
    \label{eq:equilibrium-graph}
    \div{\mathbf{s}}_v = \mathbf{a}_v - \mathbf{b}_v.
\end{equation}

Оператор дивергенції~\ref{def:div} визначає кількість <<товару>>, що виходить з вершини.
Таким чином~(\ref{eq:equilibrium-graph}) накладає наступну умову на потік: кількість трафіку з вершини має бути рівною
надлишковому попиту в цій вершині.

Тепер визначимо задачу мінімального потоку на графі.

\begin{definition}[Задача мінімального потоку на графі]
    Нехай дано неорієнтований зв'язний граф з невід'ємними вагами $G = (V, E, w)$. Необхідно знайти такий потік
    $s : E \to \mathbb{R}_+$, що функціонал
    \begin{equation}
        \label{eq:min-flow-graph}
        \sum_{e \in E} \mathbf{s}_e \cdot w_e,
    \end{equation}
    досягає мінімуму і виконується умова збереження маси~(\ref{eq:equilibrium-graph}). 

    Потік, що мініміхує функціонал~(\ref{eq:min-flow-graph}) будемо називати \textbf{мінімальним потоком на графі}.
\end{definition}

Таке формулювання збігає з неперервним аналогом задачі мінімального потоку~\ref{def:min-flow}: ми шукаємо потік, який
мінімізує сумарну інтенсивність. 

\section{Еквівалентність задачі оптимального транспортування і задачі пошуку мінімального потоку для графу}

В цій секції буде доведено основний результат роботи.

\begin{theorem}[Еквівалентність задачі оптимального транспортування і мінімального потоку для графу]
    \label{theorem:equiv}
    Нехай дано зв'язний невід'ємно зважений граф $G = (V, E, w)$ і дві міри $\mathbf a, \mathbf b$.
    Визначимо матрицю цін $\mathbf{C}$ наступним чином
    $$
        \mathbf{C}_{ij} = \text{мінімальна вага шляху між вершинами } i, j
    $$.

    Тоді виконується рівність
    \begin{equation}
        \label{eq:eq-graph}
        \min_{\mathbf{P} \in \left(\mathbf a, \mathbf b\right)} \sum_{i, j} \mathbf{P}_{ij} \cdot \mathbf{C}_{ij} =
        \min \left\{
            \sum_{e \in E} \mathbf{s}_e \cdot w_e : \mathbf{s}_e \text{ задовольняє }\ref{eq:equilibrium-graph}
            \right\}.
    \end{equation}
\end{theorem}
\begin{proof}
    Для доведення рівності покажемо, що для довільного плану транспортування можна отримати потік,
    що задовольняє умові~(\ref{eq:equilibrium-graph}), і з довільного
    адекватного потоку можна отримати план транспортування з однаковими цінами.
    Перше покаже, що оптимальна ціна транспортування не менша, за
    ціну оптимального потоку; друге покаже, що ціна оптимального потоку не менша за ціну оптимального транспортування.

    Позначимо $\mathbf{P} \in U\left(\mathbf{a}, \mathbf{b}\right)$ деякий транспортний план, а
    через $C_{uv}$ -- множина ребер, що сполучають вершини $u, v \in V$ і сума
    $$
    \sum_{e \in C_{uv}} w_e
    $$
    мінімальна. Тобто, $C_{uv}$ -- шлях мінімальної ваги між $u, v \in V$.

    Покладемо $\mathbf{s} = \mathbf{0}$. Наступна процедура перетворить $\mathbf{s}$ в потік,
    що задовольняє~(\ref{eq:equilibrium-graph}):
    \begin{enumerate}
        \item Для кожної пари вершин $(u, v) \in V^2$, якщо значення $\mathbf{P}_{uv}$ більше 0, знайти шлях мінімальної довжини 
        $C_{uv}$
        \item Для кожного ребра $e \in C_{uv}$ збільшити значення $\mathbf{s}_e$ на величину $\mathbf{P}_{e}$
    \end{enumerate}
    Визначимо ціну такого потоку:
    $$
        \sum_{e \in E} \mathbf{s}_e \cdot w_e = \sum_{(u, v) \in E} \mathbf{s}_{uv} \cdot w_{uv}.
    $$
    Для цього розпишемо $\mathbf{s}_{uv}$. Отримаємо:
    $$
        \sum_{(u, v) \in E} w_{uv} \cdot \left(\sum_{(i, j) \in V^2 : (u, v) \in C_{ij}} \mathbf{P}_{ij}\right).
    $$
    Змінимо порядок підсумовування:
    $$
        \sum_{(i, j) \in V^2} \mathbf{P}_{ij} \cdot 
        \left(
            \sum_{(u, v) \in E : (u, v) \in C_{ij}} w_{uv}
        \right).
    $$
    Помітимо, що
    $$
        \mathbf{C}_{ij} = \sum_{(u, v) \in E : (u, v) \in C_{ij}} w_{uv}.
    $$
    за означенням матриці $\mathbf{C}$.
    Таким чином
    $$
        \sum_{e \in E} \mathbf{s}_e \cdot w_e = 
        \sum_{(i, j) \in V^2} \mathbf{P}_{ij} \cdot 
        \left(
            \sum_{(u, v) \in E : (u, v) \in C_{ij}} w_{uv}
        \right) = \sum_{(i, j) \in V^2} \mathbf{P}_{ij} \cdot \mathbf{C}_{ij}.
    $$
    Звідки
    $$
        \min_{\mathbf{P} \in \left(\mathbf a, \mathbf b\right)} \sum_{i, j} \mathbf{P}_{ij} \cdot \mathbf{C}_{ij} \ge
        \min \left\{
            \sum_{e \in E} \mathbf{s}_e \cdot w_e : \mathbf{s}_e \text{ задовольняє }(\ref{eq:equilibrium-graph})
            \right\}.
    $$
    Доведемо зворотню нерівність. Для цього побудуємо транспортний план з потоку.

    Нехай $\mathbf{s}$ -- потік, що задовольняє умові~(\ref{eq:equilibrium-graph}). Визначимо орієнтований граф
    $G_s = (V_s, E_s)$, де $V_s = \{v \in V: \exists u \in V: \mathbf{s}_{(u, v)} \ge 0 \lor \mathbf{s}_{(v, u)} \ge 0\}$,
    $E_s = \{e \in E : \mathbf{s}_e \ge 0\}$.
    Тобто граф $G_s$ -- це граф, що складається з усіх вершин і ребер, які застосовуються оптимальним потоком.

    Визначимо множини джерел $S$ та стоків $D$ як
    \begin{eqnarray*}
        S = \{v \in V: \mathbf{a}_v - \mathbf{b}_v < 0\}; \\ 
        D = \{v \in V: \mathbf{a}_v - \mathbf{b}_v > 0\}.
    \end{eqnarray*}

    Нехай $\mathbf{s}$ -- деякий коректний потік. Визначимо нульову матрицю $\mathbf{P}$ розміру $n \times n$.
    \begin{enumerate}
        \item Поки $\mathbf{s} \neq \mathbf{0}$;
        \item Для всіх вершин $s \in S$, для всіх вершин $d \in D$;
        \item Знайти найкоротший шлях $C_{sd}$ графу $G_s$, де $\forall e \in C_{sd} > 0$;
        \item Визначити значення $\delta_{sd} := \min_{e \in C_{sd}} \mathbf{s}_e$;
        \item Додати $\delta$ до $\mathbf{P}_{sd}$;
        \item Для всіх $e \in C_{sd}$ зменшити $\mathbf{s}_e$ на $\delta$.
    \end{enumerate}

    Визначимо ціну такого транспортного плану. Будемо вважати, що під час роботи алгоритму на кроці 3 було виявлено $k$
    шляхів; позначимо $k$-й шлях $C^k$. Таким чином, ціна транспортування рівна
    $$
        \sum_{k} \delta_k  \cdot \sum_{(i, j) \in C^k} w_{ij} = \sum_{k} \sum_{(i, j) \in C^k} (\delta_k \cdot w_{ij}). 
    $$
    Змінимо порядок підсумовування:
    $$
        \sum_{(i, j) \in E_s} \sum_{k: (i, j) \in C^k} (\delta_k \cdot w_{ij}) = \sum_{(i, j) \in E_s} w_{ij}
        \left(
            \sum_{k:(i, j) \in C^k} \delta_k
        \right).
    $$
    Помітимо, що для довільного ребра $(i, j) \in E_s$ вираз 
    \begin{equation}
        \label{eq:some-eq}
        \sum_{k:(i, j) \in C^k} \delta_k,
    \end{equation}
    підраховує $\mathbf{s}_{ij}$. Дійсно, процедура описана вище працює доки потік $\mathbf{s}$ не вичерпається.
    Таким чином, вираз~(\ref{eq:some-eq}) своего роду є <<розшаруванням>> значення $\mathbf{s}_{ij}$.
    Звідки
    \begin{align*}
        &\sum_{(i, j) \in E_s} \sum_{k: (i, j) \in C^k} (\delta_k \cdot w_{ij}) = \sum_{(i, j) \in E_s} w_{ij}
        \left(
            \sum_{k:(i, j) \in C^k} \delta_k
        \right) = \\
        = \sum_{(i, j) \in E_s} w_{ij}\cdot \mathbf{s}_{ij}.
    \end{align*}
    Отже
    $$
        \sum_{i, j} \mathbf{P}_{ij} \cdot \mathbf{C}_{ij} = \sum_{k} \delta_k  \cdot \sum_{(i, j) \in C^k} w_{ij} = 
        \sum_{(i, j) \in E_s} w_{ij} \cdot \mathbf{s}_{ij}.
    $$
    Це доводить нерівність
    $$
        \min_{\mathbf{P} \in \mathbf{U}\left(\mathbf a, \mathbf b\right)} \sum_{i, j} \mathbf{P}_{ij} \cdot \mathbf{C}_{ij} \le
        \min \left\{
            \sum_{e \in E} \mathbf{s}_e \cdot w_e : \mathbf{s}_e \text{ задовольняє }(\ref{eq:equilibrium-graph})
            \right\}.
    $$
    звідки отримуємо твердження теореми.
\end{proof}

Теорему~\ref{theorem:equiv} можна розглянути з іншого боку: нехай кожна вершина $v \in V$ -- це частка з деякою масою
$\mathbf{a}_v - \mathbf{b}_v$, а $e \in E$ -- <<елементарний об'єм>>. В такому разі пошук оптимального плану, це визначення
руху індивідуальної частки $v \in V$ по <<всьому об'єму>>,
а визначення оптимального потоку -- це визначення загальної маси часток, що проходить по конкретному
<<елементарному об'єму>> -- ребру $e \in E$.

Подібні підходи відомі в гідродинаміці (і теорії поля загалом) як \textit{лагранжевий} та \textit{ейлерів} підхід відповідно. 
Обидва підходи є еквівалентними і пов'язані рівнянням непервності, що де-факто є аналогом рівняння збереження маси,
аналогічно до~(\ref{eq:equilibrium-graph}).

\chapconclude{\ref{chap:graph}}
У цьому розділі було надано визначення аналогів диверенціальних операторів, за допомогою яких
було розроблено формулювання задачі Канторовича~(\ref{def:kantorovich-problem})
та задачі мінімального потоку~(\ref{def:min-flow}) на випадок графів.

Було конструктивно доведено еквівалентність задач на графу, шляхом надання відповідних процедур перетворення
транспортного плану на потік і навпаки.

Наступний розділ буде присвячено формулюванню понять, що дозволяють чисельно перевірити коретність, та безпосередньо
перевірці результатів поточного розділу.