% Титульный лист
\linespread{1.1}

\begin{center}
{\bfseries
НАЦІОНАЛЬНИЙ ТЕХНІЧНИЙ УНІВЕРСИТЕТ УКРАЇНИ \par
<<КИЇВСЬКИЙ ПОЛІТЕХНІЧНИЙ ІНСТИТУТ \par
імені Ігоря СІКОРСЬКОГО>>\par
НАВЧАЛЬНО-НАУКОВИЙ ФІЗИКО-ТЕХНІЧНИЙ ІНСТИТУТ\par
Кафедра математичного моделювання та аналізу даних}
\end{center}
\par

\linespread{1.1}
Рівень вищої освіти --- перший (бакалаврський)

Спеціальність (освітня програма) --- 113~Прикладна математика,

ОПП <<Математичні методи моделювання, розпізнавання образів та комп'ютерного зору>>

\vspace{10mm}
\begin{tabularx}{\textwidth}{XX}
& ЗАТВЕРДЖУЮ                              \\[06pt]
& В.о. завідувача кафедри                 \\[06pt]
& \rule{2.5cm}{0.25pt} Іван ТЕРЕЩЕНКО     \\[06pt]
& <<\rule{0.5cm}{0.25pt}>> \rule{2.5cm}{0.25pt} \YearOfDefence~р. 
\end{tabularx}

\vspace{5mm}
\begin{center}
{\bfseries ЗАВДАННЯ \par}
{\bfseries на дипломну роботу \par}
\end{center}

%%%%%====================================
% !!! Не чіпайте наступні три команди!
%%%%%====================================
\frenchspacing
\doublespacing          % інтервал "1,5" між рядками, тепер навічно
\setfontsize{14}

Студент: \underline{\reportAuthor} \par

1. Тема роботи: <<\emph{\reportTitle}>>,

керівник: \underline{\supervisorRegalia ~\supervisorFio}, \par
затверджені наказом по університету \No \underline{1761-с} від <<\underline{26}>> \underline{05} \YearOfDefence~р.

2. Термін подання студентом роботи: <<\rule{0.5cm}{0.25pt}>> \rule{2.5cm}{0.25pt} \YearOfDefence~р.

3. Вихідні дані до роботи: послідовність кроків алгоритму побудови оптимального транспортування на графах. 

4. Зміст роботи: 
\begin{enumerate}[label=\arabic*)]
    \item Провести огляд джерел за тематикою дослідження;
    \item Довести еквівалентність задач на графі;
    \item Перевірити результати експериментально.
\end{enumerate} 

5. Перелік ілюстративного матеріалу: презентація доповіді

6. Дата видачі завдання: 10 вересня \YearOfBeginning~р.

% Якщо перша частина завдання вилізе за сторінку - приберіть команду \newpage
% Календарний план є продовженням завдання, а не окремою частиною
\newpage

\begin{center}
Календарний план
\end{center}

\renewcommand{\arraystretch}{1.5}
\begin{table}[h!]
\setfontsize{14pt}
\centering
    \begin{tabularx}{\textwidth}{|>{\centering\arraybackslash\setlength\hsize{0.25\hsize}}X|>{\setlength\hsize{2\hsize}}X|>{\centering\arraybackslash\setlength\hsize{1\hsize}}X|>{\centering\arraybackslash\setlength\hsize{0.75\hsize}}X|}
    \hline \No\par з/п & Назва етапів виконання дипломної роботи & Термін виконання & Примітка \\
    \hline 
    % номер етапу
    1 & 
    % назва етапу
    Узгодження теми роботи із науковим керівником & 
    % термін виконання
    01-15 вересня \YearOfBeginning~р. &
    % примітка - зазвичай "Виконано"
    Виконано \\
%%% -- початок інтервалу для копіювання
    \hline 
    % номер етапу
    2 & 
    % назва етапу
    Огляд опублікованих джерел за тематикою оптимального транспортування & 
    % термін виконання
    Вересень-жовтень \YearOfBeginning~р. &
    % примітка - зазвичай "Виконано"
    Виконано \\
%%% -- початок інтервалу для копіювання
    \hline
    3 &
    Огляд опублікованих джерел за тематикою задачі Бекмана і її аналогів &
    Жовтень-грудень \YearOfBeginning~р. & Виконано \\
%%% -- початок інтервалу для копіювання
    \hline
    4 &
    Отримання узагальнень на випадок не орієнтованих невід'ємно зважених графів &
    Січень-лютий \YearOfDefence~р. & Виконано \\
%%% -- початок інтервалу для копіювання
    \hline
    5 &
    Доведення еквівалентності задач і побудова процедури перетворення потоку на траспортний план &
    Лютий-Березень \YearOfDefence~р. & Виконано \\
%%% -- початок інтервалу для копіювання
    \hline
    6 &
    Розробка програмного забезпечення мовою Python & 
    Квітень \YearOfDefence~р. & Виконано \\
%%% -- початок інтервалу для копіювання
    \hline
    7 &
    Оформлення дипломної роботи &
    Травень \YearOfDefence~р. & Виконано \\



%скопійовані інтервали вставляти перед фінальною \hline та заповнювати відповідно
    \hline %фінальна hline
    \end{tabularx}
\end{table}

\renewcommand{\arraystretch}{1}
\begin{tabularx}{\textwidth}{>{\setlength\hsize{1.5\hsize}}X >{\setlength\hsize{0.5\hsize}}X >{\setlength\hsize{1\hsize}}X}
Студент  & \rule{2.5cm}{0.25pt}  & \reportAuthorShort \\[06pt]
Керівник & \rule{2.5cm}{0.25pt}  & \supervisorFioShort     \\
\end{tabularx}

\newpage
