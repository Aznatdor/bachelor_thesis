% !TeX program = pdflatex
% !BIB TS-program = biber

% ********** Приклад оформлення пояснювальної записки **********
% *********  до атестаційної роботи ступеня бакалавра **********


\documentclass{bachelor_thesis}

% Додаткові пакети вносіть у цей файл
% !TeX root = thesis.tex
\usepackage{minted}
\usepackage{xcolor}
\usepackage{listings}
\usepackage[most]{tcolorbox}
\usepackage{enumitem}
\usepackage{csquotes}
\usepackage{bbold}
\usepackage{tabularray}
\usepackage{tikz}
\usepackage{ifthen}

\usetikzlibrary{positioning}

\tcbset{
  colback=gray!5!white,
  colframe=gray!50!black,
  fonttitle=\bfseries,
  boxrule=0.5pt,
  arc=2mm,
  left=1em,
  right=1em,
  top=0.5em,
  bottom=0.5em,
  boxsep=0pt
}

% Інший стиль - і ніфіга працювати не буде
\usepackage[backend=biber,style=gost-numeric,language=auto,autolang=other]{biblatex}
\addbibresource{thesis.bib} 
\AtBeginBibliography{\setfontsize{14}}

% Додаткові визначення та перевизначення команд вносіть у цей файл
% !TeX root = thesis.tex
\def\lip#1{\|#1\|_{\text{Lip}}} % Ліпшецева стала

\def\div#1{\text{div}\left(#1\right)} % Оператор дивіргенції

% Функція для малювання повного графа
\newcommand{\completegraph}[1]{%
    \begin{tikzpicture}[scale=3]
        \def\n{#1}
        \def\radius{1}
        % Малюємо вершини по колу
        \foreach \i in {1,...,\n} {
            \pgfmathsetmacro\angle{360/\n * (\i - 1)}
            \node[draw, circle, inner sep=1pt] (v\i) at ({\radius*cos(\angle)}, {\radius*sin(\angle)}) {};
        }
        % З'єднуємо усі вершини
        \foreach \i in {1,...,\n} {
            \foreach \j in {1,...,\n} {
                \ifthenelse{\i < \j}{
                    \draw (v\i) -- (v\j);
                }{}
            }
        }
    \end{tikzpicture}
}

% Бюрократичні відомості про автора роботи
%%% Основні відомості %%%
\newcommand{\reportAuthor}             % ПІБ автора
{Бобров Андрій Олексійович}
\newcommand{\reportAuthorShort}             % ПІБ автора
{Бобров А.О.}
\newcommand{\reportAuthorGroup}        % група автора
{ФІ-12}
\newcommand{\reportTitle}              % Назва роботи
{Оптимальне траснспортування на графах і задача Бекмана}
%% використовуйте символ "\par" або "\\" для розбиття назви на декілька рядків

\newcommand{\supervisorFio}            % Науковий керівник, ПІБ повністю
{Хайдуров Владислав Володимирович}
\newcommand{\supervisorFioShort}
{Хайдуров В. В.}
\newcommand{\supervisorRegalia}        % Науковий керівник: звання, степінь, посада
{ст. досл., к.т.н., доц.}

\newcommand{\consultFio}               % Консультант, ПІБ повністю
{Рябов Георгій Валентинович}
\newcommand{\consultRegalia}           % Консультант: звання, степінь, посада
{канд. фіз.-мат. н., снс.}             % Звання нема
% Якщо у вас нема консультанта - залишайте ці поля порожніми


\newcommand{\reviewerFio}              % Рецензент, ПІБ повністю
{Ніщенко Ірина Іванівна}                        
\newcommand{\reviewerRegalia}          % Рецензент: звання, степінь, посада
{доц., канд. фіз.-мат. н.,}

\newcommand{\YearOfDefence}            % рік захисту
{2025}
\newcommand{\YearOfBeginning}          % попередній рік - може, можна це якось автоматизувати, нє?
{2024}


% Починаємо верстку документа
\begin{document}

\pagestyle{plain}
\setfontsize{14}

% Створюємо титульну сторінку
% Титульный лист
\thispagestyle{empty}
\linespread{1.1}
\setfontsize{14pt}
\begin{center}
{\bfseries
НАЦІОНАЛЬНИЙ ТЕХНІЧНИЙ УНІВЕРСИТЕТ УКРАЇНИ \par
<<КИЇВСЬКИЙ ПОЛІТЕХНІЧНИЙ ІНСТИТУТ \par
імені Ігоря СІКОРСЬКОГО>>\par
НАВЧАЛЬНО-НАУКОВИЙ ФІЗИКО-ТЕХНІЧНИЙ ІНСТИТУТ\par
\medskip
Кафедра математичного моделювання та аналізу даних}
\end{center}

\vspace{10mm}

\begin{tabularx}{\textwidth}{XX}
& <<До захисту допущено>> \\
& В.о. завідувача кафедри \\
& \rule{1.8cm}{0.25pt} Іван ТЕРЕЩЕНКО \\
& <<\rule{0.8cm}{0.25pt}>> \rule{3.2cm}{0.25pt} \YearOfDefence~р. 
\end{tabularx}

\linespread{1.5}                    % Одинарный интервал
\begin{center}
\vspace{10mm}
{\bfseries\huge Дипломна робота \par}
{\bfseries на здобуття ступеня бакалавра \par}
\end{center}

зі спеціальності: 113 Прикладна математика \par
на тему: \textbf{<<\reportTitle>>}

\vspace{10mm}

\begin{tabularx}{\textwidth}{>{\setlength\hsize{1.5\hsize}}X >{\setlength\hsize{0.3\hsize}}X}
Виконав: студент \underline{~4~} курсу, групи \underline{\reportAuthorGroup} & \\
\underline{\reportAuthor} & \rule{2.5cm}{0.25pt} \\
\vspace{5mm}
Керівник: \underline{\supervisorRegalia} & \\
\underline{\supervisorFio} & \rule{2.5cm}{0.25pt} \\
\vspace{5mm}
Консультант: \underline{\consultRegalia} & \\
\underline{\consultFio} & \rule{2.5cm}{0.25pt} \\
\vspace{5mm}


Рецензент: 
\underline{\reviewerRegalia} & \\
\underline{\text{доцент кафедри ММЗІ}} & \\
\underline{\reviewerFio} & \rule{2.5cm}{0.25pt} \\
\end{tabularx}

\vspace{15mm}

\linespread{1.1}                    % Одинарный интервал
\begin{tabularx}{\textwidth}{>{\setlength\hsize{1.25\hsize}}X >{\setlength\hsize{1.5\hsize}}X >{\setlength\hsize{0.25\hsize}}X}
& Засвідчую, що у цій дипломній роботі немає запозичень з праць інших 
авторів без відповідних посилань.

& \\
& Студент \rule{2.5cm}{0.25pt}      &
\end{tabularx}

%\vspace{10mm}
\vfill
\begin{center}
{Київ~---~\YearOfDefence}
\end{center}

\newpage
\thispagestyle{plain}

% Створюємо завдання
% Титульный лист
\linespread{1.1}

\begin{center}
{\bfseries
НАЦІОНАЛЬНИЙ ТЕХНІЧНИЙ УНІВЕРСИТЕТ УКРАЇНИ \par
<<КИЇВСЬКИЙ ПОЛІТЕХНІЧНИЙ ІНСТИТУТ \par
імені Ігоря СІКОРСЬКОГО>>\par
НАВЧАЛЬНО-НАУКОВИЙ ФІЗИКО-ТЕХНІЧНИЙ ІНСТИТУТ\par
Кафедра математичного моделювання та аналізу даних}
\end{center}
\par

\linespread{1.1}
Рівень вищої освіти --- перший (бакалаврський)

Спеціальність (освітня програма) --- 113~Прикладна математика,

ОПП <<Математичні методи моделювання, розпізнавання образів та комп'ютерного зору>>

\vspace{10mm}
\begin{tabularx}{\textwidth}{XX}
& ЗАТВЕРДЖУЮ                              \\[06pt]
& В.о. завідувача кафедри                 \\[06pt]
& \rule{2.5cm}{0.25pt} Іван ТЕРЕЩЕНКО     \\[06pt]
& <<\rule{0.5cm}{0.25pt}>> \rule{2.5cm}{0.25pt} \YearOfDefence~р. 
\end{tabularx}

\vspace{5mm}
\begin{center}
{\bfseries ЗАВДАННЯ \par}
{\bfseries на дипломну роботу \par}
\end{center}

%%%%%====================================
% !!! Не чіпайте наступні три команди!
%%%%%====================================
\frenchspacing
\doublespacing          % інтервал "1,5" між рядками, тепер навічно
\setfontsize{14}

Студент: \underline{\reportAuthor} \par

1. Тема роботи: <<\emph{\reportTitle}>>,

керівник: \underline{\supervisorRegalia ~\supervisorFio}, \par
затверджені наказом по університету \No \underline{1761-с} від <<\underline{26}>> \underline{05} \YearOfDefence~р.

2. Термін подання студентом роботи: <<\rule{0.5cm}{0.25pt}>> \rule{2.5cm}{0.25pt} \YearOfDefence~р.

3. Вихідні дані до роботи: послідовність кроків алгоритму побудови оптимального транспортування на графах. 

4. Зміст роботи: 
\begin{enumerate}[label=\arabic*)]
    \item Провести огляд джерел за тематикою дослідження;
    \item Довести еквівалентність задач на графі;
    \item Перевірити результати експериментально.
\end{enumerate} 

5. Перелік ілюстративного матеріалу: презентація доповіді

6. Дата видачі завдання: 10 вересня \YearOfBeginning~р.

% Якщо перша частина завдання вилізе за сторінку - приберіть команду \newpage
% Календарний план є продовженням завдання, а не окремою частиною
\newpage

\begin{center}
Календарний план
\end{center}

\renewcommand{\arraystretch}{1.5}
\begin{table}[h!]
\setfontsize{14pt}
\centering
    \begin{tabularx}{\textwidth}{|>{\centering\arraybackslash\setlength\hsize{0.25\hsize}}X|>{\setlength\hsize{2\hsize}}X|>{\centering\arraybackslash\setlength\hsize{1\hsize}}X|>{\centering\arraybackslash\setlength\hsize{0.75\hsize}}X|}
    \hline \No\par з/п & Назва етапів виконання дипломної роботи & Термін виконання & Примітка \\
    \hline 
    % номер етапу
    1 & 
    % назва етапу
    Узгодження теми роботи із науковим керівником & 
    % термін виконання
    01-15 вересня \YearOfBeginning~р. &
    % примітка - зазвичай "Виконано"
    Виконано \\
%%% -- початок інтервалу для копіювання
    \hline 
    % номер етапу
    2 & 
    % назва етапу
    Огляд опублікованих джерел за тематикою оптимального транспортування & 
    % термін виконання
    Вересень-жовтень \YearOfBeginning~р. &
    % примітка - зазвичай "Виконано"
    Виконано \\
%%% -- початок інтервалу для копіювання
    \hline
    3 &
    Огляд опублікованих джерел за тематикою задачі Бекмана і її аналогів &
    Жовтень-грудень \YearOfBeginning~р. & Виконано \\
%%% -- початок інтервалу для копіювання
    \hline
    4 &
    Отримання узагальнень на випадок не орієнтованих невід'ємно зважених графів &
    Січень-лютий \YearOfDefence~р. & Виконано \\
%%% -- початок інтервалу для копіювання
    \hline
    5 &
    Доведення еквівалентності задач і побудова процедури перетворення потоку на траспортний план &
    Лютий-Березень \YearOfDefence~р. & Виконано \\
%%% -- початок інтервалу для копіювання
    \hline
    6 &
    Розробка програмного забезпечення мовою Python & 
    Квітень \YearOfDefence~р. & Виконано \\
%%% -- початок інтервалу для копіювання
    \hline
    7 &
    Оформлення дипломної роботи &
    Травень \YearOfDefence~р. & Виконано \\



%скопійовані інтервали вставляти перед фінальною \hline та заповнювати відповідно
    \hline %фінальна hline
    \end{tabularx}
\end{table}

\renewcommand{\arraystretch}{1}
\begin{tabularx}{\textwidth}{>{\setlength\hsize{1.5\hsize}}X >{\setlength\hsize{0.5\hsize}}X >{\setlength\hsize{1\hsize}}X}
Студент  & \rule{2.5cm}{0.25pt}  & \reportAuthorShort \\[06pt]
Керівник & \rule{2.5cm}{0.25pt}  & \supervisorFioShort     \\
\end{tabularx}

\newpage


% У даному костильному рішенні перші три сторінки (титул та завдання на 
% роботу) друкуються окремо від основної частини тез.
% Тому перша сторінка сформованого документу нумерується як четверта

% Створюємо анотації
\setcounter{page}{4}
\abstractUkr
Квалiфiкацiйна робота мiстить: 44 сторiнки, 2 рисунки, 2 таблиці,
9 джерел.

У цій роботі було досліджено зв'язок задачі мінімального потоку (задачі Бекмана) із
задачею оптимального транспортування на випадку дискретного простору (зв'язного невід'ємно зваженого графу).

В ході дослідження було доведено еквівалентність задачі мінімального потоку та задачі оптимального транспортування
на графі. Було отримано алгоритм побудови оптимального транспортного плану із оптимального потоку та доведено коректність алгоритму.
Було наведено реалізацію алгоритму мовою Python. Також, було проведено порівняльний аналіз результатів роботи алгоритму з 
іншими методами побудови оптимального транспотного плану.

\MakeUppercase{ГРАФ, ПОТІК НА ГРАФІ, ОПТИМАЛЬНИЙ ТРАНСПОРТНИЙ ПЛАН}

\abstractEng
The qualification work contains: 44 pages, 2 figures, 2 tables, and 9 citations.

This work investigates the relationship between the minimum flow problem (the Beckmann problem) and 
the optimal transport problem in the case of a discrete space (a connected non-negatively weighted graph).

In the course of the research, the equivalence of the minimum flow problem and the optimal transport problem on a graph was proven. An algorithm for constructing an optimal transport plan from an optimal flow was obtained, and the correctness of the algorithm was proven.
An implementation of the algorithm in Python was presented. A comparative analysis of the algorithm's results with other
methods of constructing the optimal transport plan was also conducted.

\MakeUppercase{GRAPH, FLOW ON A GRAPH, OPTIMAL TRANSPORTATION PLAN}
\clearpage

% Створюємо зміст
\pagenumbering{gobble}
\tableofcontents
\cleardoublepage
\pagenumbering{arabic}
\setcounter{page}{7}    %!!! -- продумати, як автоматизувати номер сторінки

% Створюємо перелік умовних позначень, скорочень і термінів
% Якщо цей розділ вам не потрібен, просто закоментуйте два наступних рядка
\shortings
%!TEX root = ../thesis.tex
% ===================== Запхнути це в оточення array ==========================
\begin{tblr}{}
    &$\mathbb{R}_+$  &--- &множина невід'ємних дійсних чисел. \\

    &$\delta_A$      &--- &міра Дірака множини $A$. \\

    &$C_{uv}$        &--- &шлях мінімальної ваги між вершинами $u, v$. \\

    &$\mathbf{n}$    &--- &вектор зовнішньої нормалі. \\

    &$\partial A$    &--- &межа множини $A$. \\

    &$d H^n$         &--- &$n$-вимірна міра Хаусдорфа. \\

    &$d\lambda^n$    &--- &$n$-вимірна міра Лебега. \\

    &$\mu[A]$        &--- &значення міри $\mu$ на множині $A$. \\

    &$\Pi(\mu, \nu)$ &--- &множина каплінгів мір $\mu$ та $\nu$. \\

    &$\otimes$       &--- &добуток Кронекера. \\

    &$L^1(\mu)$      &--- &множина абсолютно інтегровних функцій \\
    &&&відносно міри $\mu$. \\

    &$\lip{\varphi}$ &--- &скорочене позначення сталої Ліпшиця \\
    &&&$\sup_{x \neq y} \dfrac{|\varphi(x) - \varphi(y)|}{d(x, y)}$. \\

    &$\varphi^c$     &--- &c-перетворення функції $\varphi: X \to \mathbb{R}$ визначене як \\
    &&&$\varphi^c(y) = \inf_{x \in X}[c(x, y) - \varphi(x)]$. \\

    &$\varphi^{cc}$  &--- &cc-перетворення функції $\varphi: X \to \mathbb{R}$ визначене як \\
    &&&$\varphi^{cc}(x) = \inf_{y \in Y}[c(x, y) - \varphi^c(y)]$.
\end{tblr}

% Створюємо вступ
\intro
%!TEX root = ../thesis.tex

\textbf{Актуальність дослідження.} Актуальність даного дослідження полягає в широкому використанні графу для моделювання
зв'язків між містами і побудови транспортних маршрутів в логістиці. В загальному випадку побудова оптимальних транспортних
маршрутів через розв'язання лінійних задач не є оптимальною за часом. Мотивацією для запропонованого в цій роботі підходу
є те, що побудова оптимального потоку є доволі швидкою процедурою і вимагає менше пам'яті.

\textbf{Метою дослідження} є побудова і реалізація алгоритму перетворення потоку на графі на план транспортування.

\begin{enumerate}
    \item провести огляд опублікованих джерел за тематикою дослідження;
    \item довести еквівалентність поставлених задач;
    \item отримати процедуру перетворення оптимального потоку на оптимальний транспортний план;
    \item перевірити одержані результати експериментально.
\end{enumerate}

\textit{Об'єктом дослідження} є логістично-економічні процеси.

\textit{Предметом дослідження} є математичні методи, графові моделі і алгоритми.

При розв'язанні поставлених задач використовувались такі \textit{методи дослідження}: методи дискретної математики (теорія графів),
функціонального аналізу (теорії міри) та комп'ютерного моделювання (проведення обчислювальних експериментів).

\textbf{Практичне значення} полягає в прискорені побудови оптимального транспортного плану на графі.

% Додаємо глави
% Якщо ваша робота містить менше або більше глав - модифікуйте наступні 
% рядки відповідним чином
%!TEX root = ../thesis.tex
\chapter{Задача оптимального транспортування та задача пошуку оптимального потоку}
\label{chap:review}
В першому розділі розглянуто поняття задачі оптимального транспортування та 
поняття задачі оптимального потоку для неперервного випадку.
Розглянуто два підходи до формулювання задачі оптимального транспортування, і зв'язок між ними.

\section{Задача оптимального транспортування}
Сформулюємо основні поняття теорії оптимального транспортування.

\begin{definition}[Образ міри при відображенні]
    \label{def:push-forward}
    Нехай задана міра $\mu$ на вимірному просторі $X$ і вимірне відображення $T : X \to Y$.
    \textbf{Образом міри $\mu$ при відображені $T$} називається міра $\nu$ на $Y$, для якої справедливо:
    $$
        \forall A \subset Y, A\text{ -- вимірна}: \nu[A] = \mu\left[T^{-1}(A)\right]
    $$
    Позначають: $\nu = T \# \mu$.
\end{definition}


\begin{definition}
    \label{def:coupling}
    Нехай дана міра $\mu$ на множині $X$ і міра $\nu$ на множині $Y$. \textbf{Каплінгом} мір $\mu$ та $\nu$ називається
    міра $\pi$ на $X \times Y$ така, що для довільних вимірнірних множин $A \subset X$ та $B \subset Y$ виконується
    \begin{eqnarray}
        \pi[A \times Y] = \mu[A]; \\
        \pi[X \times B] = \nu[B].
    \end{eqnarray}
\end{definition}

Множину усіх каплінгів мір $\mu$ та $\nu$ будемо позначати $\Pi(\mu, \nu)$.

\subsection{Задача Монжа}
Гаспар Монж у своїй роботі 1781 року~\cite{monge1781} сформулював задачу оптимального перенесення купи ґрунту $X$ 
в яму $Y$ того ж об'єму. Перенесення визначається за допомогою деякої функції $T : X \to Y$, що не формально
позначає в яку точку ями $y \in Y$ треба перевезти ґрунт з точки $x \in X$.

Сучасне формулювання задачі Монжа має наступний вигляд:

\begin{definition}[Задача Монжа]
    \label{def:monge-problem}
    Дано розподіли $\mu$ на $X$ та $\nu$ на $Y$ і деяка функція ціни $c : X \times Y \to \mathbb{R}_+ \cup \{+\infty\}$,
    яка визначає вартість перевезення ґрунту з точки $x \in X$ в точку $y \in Y$. 
    Необхідно знайти функцію $T : X \to Y$ таку, що $T \# \mu = \nu$ і функціонал
    $$
        \int_X c\left(x, T(x)\right)\, d\mu(x),
    $$
    досягає мінімуму.
\end{definition}

Основною проблемою такого формулювання є той факт, що маса деякої точки $x \in X $не може бути розділена між двома місцями
в ямі $y_1, y_2 \in Y$.

\subsection{Задача Канторовича}
\label{sec:kantorovich-problem}

У 1942 Леонід Канторович, незалежно від Монжа, сформулював задачу транспортування товару до споживачів~\cite{kantorovich2006}.

У формалізації Канторовича ми маємо дві множини $X$ та $Y$, що відповідно є множинами <<товару>> та <<споживачів>>;
щільності <<виробництва товару>> і <<споживання>> відповідно рівні $\mu$ та $\nu$.
Задача полягає в пошуку оптимального плана перевезання $\pi \in \Pi(\mu, \nu)$ усього <<товару>> до <<споживачів>>, за умови, що
ціна перевезення одиниці товару з точки $x \in X$ до споживача у точці $y \in Y$ рівна $c(x, y)$.

\begin{definition}[Задача Канторовича]
    \label{def:kantorovich-problem}
    Дано розподіли $\mu$ на $X$ та $\nu$ на $Y$ і деяка функція ціни $c : X \times Y \to \mathbb{R}_+ \cup \{+\infty\}$. 
    Необхідно зайти міру $\pi \in \Pi(\mu, \nu)$, яка мінімізує функціонал
    $$
        \int_{X \times Y} c(x, y)\, d\pi,
    $$
\end{definition}

На сьогоднішній день формулювання вище має назву \textit{задачі Монжа-Канторовича}.

\subsection{Дуальність}
Задача Канторовича має дуальне формулювання.

Визначимо множину:
$$
    \Phi_c = \{(\varphi, \psi) \in L^1(\mu) \times L^1(\nu) : \forall (x, y) \in X \times Y : \varphi(x) + \psi(y) \le c(x, y)\},
$$
і визначимо функціонал $J(\varphi, \psi)$ як:
$$
    J(\varphi, \psi) = \int_X \varphi\, d\mu + \int_Y \psi\, d\nu.
$$
Тоді справедлива наступна теорема

\begin{theorem}[Дуальність Канторовича~\cite{villani2003}]
    \label{theorem:kantorovich-duality}
    Нехай $X$ та $Y$ --- дві компактні множини з мірами $\mu$ та $\nu$, відповідно. 
    Нехай $c : X \times Y \to \mathbb{R}_+ \cup \{+\infty\}$ --- напівнеперервна знизу функція ціни. Тоді:
    $$
        \inf_{\pi \in \Pi(\mu, \nu)} \int_{X \times Y} c(x, y)\, d\pi(x, y) = \sup_{(\varphi, \psi) \in \Phi_c} J(\varphi, \psi).
    $$
\end{theorem}

Ця теорема є фундаметом для подальшого дослідження як методів розв'язку задачі Канторовича,
так і зв'язку з іншими задачами оптимізації. 

Зокрема, важливим наслідком теореми~\ref{theorem:kantorovich-duality} є наступна теорема.

\begin{theorem}[Дуальність Канторовича-Рубінштейна~\cite{villani2003}]
    \label{theorem:kr-duality}
    Нехай $(X, \rho)$ --- компактний метричний простір і задані дві міри $\mu$ та $\nu$ на $X$.
    Для функції ціни $c(x, y) = \rho(x, y)$ виконується:
    \begin{equation}
        \begin{aligned}
            \label{eq:kr-duality}
            \inf_{\pi \in \Pi(\mu, \nu)} \int_{X \times Y} \rho(x, y)\, d\pi(x, y) = \\
            = \sup_{\varphi \in L^1(\mu - \nu)} 
            \left\{
                \int_X \varphi\, d(\mu - \nu) : \lip{\varphi} \le 1
            \right\}.
        \end{aligned}
    \end{equation}
\end{theorem}

\section{Задача пошуку мінімального потоку}
\label{sec:min-flow-problem}

Нехай дана деяка достатньо гладенька і зв'язна область $\Omega \subset \mathbb{R}^2$,
яку можна інтерпретувати як деяке <<місто>>.
На множині $\Omega$ визначені дві ймовірнісні міри $\mu$, $\nu$ --- відповідно міра локального попиту 
та локальної пропозиції.
В цих позначеннях можна вважати, що міра $(\mu - \nu)$ є локальною мірою надлишкового попиту.

Будемо вважати, що трафік товару моделюється як деяке векторне поле $\mathbf{Y} : \Omega \to \mathbb{R}^2$. 
Таким чином напрям $\mathbf Y(x)$ визначає напрям руху товару в точці $x \in \Omega$,
а $\left|\mathbf{Y}(x)\right|$ визначає інтенсивність цього руху.

Природньо вважати, що виконується аналог закону збереження маси:
кількісно витік споживачів із довільної області $K \subset \Omega$ рівний надлишковому попиту в цій області:
\begin{equation*}
    \int_{\partial K} \mathbf{Y} \cdot \mathbf{n}\, dH = (\mu - \nu)(K).
\end{equation*} 
Локально це еквівалентно рівності в слабкому сенсі:
\begin{equation}
    \label{eq:equilibrium}
    \div{\mathbf{Y}} = (\mu - \nu)
\end{equation}
тобто
$$
    \int_{\partial K} \mathbf{Y} \cdot \mathbf{n}\, dH = \iint_{K} \div{Y} d\lambda^2 =
    \iint_{K} d(\mu - \nu)
$$

Також будемо вважати наше місто ізольованим:
\begin{equation}
    \label{eq:isolation}
    \mathbf{Y} \cdot \mathbf{n} = 0\text{ на }\partial \Omega.
\end{equation}

\begin{definition}[Задача пошуку мінімального потоку~\cite{cocv, beckmann52}]
    \label{def:min-flow}
    Нехай дано дві міри $\mu$ та $\nu$ на множині $\Omega$
    і деяка неспадна функція $g : \mathbb{R}_+ \to \mathbb{R}_+ \cup \{+\infty\}$.
    Необхідно знайти таке векторне поле $\mathbf{Y}$, що задовольняє умовам~(\ref{eq:equilibrium})--(\ref{eq:isolation})
    і мінімізує функціонал:
    \begin{equation}
        \label{eq:min-flow}
         \int_\Omega g\left(\left|\mathbf{Y}(\omega)\right|\right)\, d\lambda^2.
    \end{equation}
       
    Векторне поле $\mathbf{Y}^*$ на якому досягається мінімум~(\ref{eq:min-flow}) називають \textbf{мінімальним потоком}.
\end{definition}

В цій дещо спрощеній моделі ми вважаємо, що ціна транспортування одиниці товару не залежить від інтенсивності (нема заторів).
Також у якості функції $g$ оберемо функцію $g(t) = t$ для $t \in \mathbb{R}_+$; в багатьох випадках це доволі природній вибір.

Тоді задача пошуку мінімального потоку приймає вигляд
$$
    \int_{\Omega} \left|\mathbf{Y}(x)\right|\, d\lambda^2.
$$

Неформально ми шукаємо такий потік, сумарна інтенсивність мінімальна.

\section{Зв'язок задачі Канторовича з пошуком мінімального потоку}
Задача пошуку мінімального потоку~\ref{def:min-flow} шукає шляхи, якими треба транспортувати товар.
З іншого боку, задача Канторовича~\ref{def:kantorovich-problem} шукає способи перевезення одиниці товару до споживача.

Природним чином постає питання про зв'язок цих задач.

\begin{theorem}[Еквівалентність задачі пошуку мінімального потоку і задачі Монжа-Канторовича~\cite{otam}]
    \label{theorem:min-flow-kantorovich}
    Нехай $\mu, \nu$ -- дві ймовірнісні міри на $\Omega$.
    Тоді:
    $$
        \inf_{\pi \in \Pi(\mu, \nu)} \int_{\Omega^2} \rho(x, y)\, d\pi(x, y) = \inf
        \left\{
            \int_\Omega \left|\mathbf{Y}(x)\right|\, d\lambda^2 
            : \mathbf{Y} \text{ задовольняє } (\ref{eq:equilibrium})-(\ref{eq:isolation})
        \right\}.
    $$
\end{theorem}

Отже, постає питання про те, яким чином ми можемо отримати оптимальний план транспортування
для задачі Канторовича~\ref{def:kantorovich-problem}, використовуючи деякий мінімальний потік, і навпаки.

\chapconclude{\ref{chap:review}}

В цьому розділу було розглянуто основні означення теорії оптимального транспортування і визначення задачі мінімального потоку.

Було розглянуто випадок, коли ціна в задачі оптимального транспортування задається метрикою. 
В цьому випадку можна побудувати еквівалентність між задачами оптимального транспортування та мінімального потоку.

Наступний розділ буде присвяченно дослідженню еквівалентності цих задач у випадку графа і метрики, індукованої вагами цього графа.
%!TEX root = ../thesis.tex
\chapter{Оптимальне транспортування і мінімальний потік на графі}
\label{chap:graph}

В цьому розділі будуть сформульовані аналоги задачі Канторовича~\ref{def:kantorovich-problem}
та задачі пошуку мінімального потоку~\ref{def:min-flow} для графу.

Буде доведено еквівалентність цих задач. Також, буде одержано процедуру побудови
оптимального плану транспортування з мінімального потоку.

\section{Задача Канторовича на скінченній множині}
Спочатку наведемо визначення каплінгу дискретних мір. В межах цієї роботи будемо вважати,
що міра на множині з $n < \infty$ елементів задається вектором з невід'ємними компонентами.

\begin{definition}[Каплінг дискретних мір]
    Нехай дано дві дискретні міри $\mathbf a \in \mathbb{R}_+^n$ та $\mathbf b \in \mathbb{R}_+^m$. Також покладемо
    $\mathbb{1}_n$ -- вектор-стовпчик одиниць.

    \textbf{Каплінгом} двох дискретних мір називається матриця $\mathbf{P} \in \mathbb{R}_{+}^{n \times m}$ така, що
    \begin{eqnarray*}
        \mathbf{P}\mathbb{1}_m = \mathbf a; \\
        \mathbf{P}^\top \mathbb{1}_n = \mathbf b
    \end{eqnarray*}
\end{definition}

Множину усіх каплінгів двох дискретних мір $\mathbf a, \mathbf b$ будемо позначати $\mathbf{U}\left(\mathbf a, \mathbf b\right)$

Тепер наведемо означення задачі Канторовича на випадок двох скінченних множин.

\begin{definition}[Лінійна програма Канторовича~\cite{cot}]
    \label{def:kantorovich-problem-d}
    Нехай дано дві скінченні множини $X$ та $Y$, причому $|X| = n, |Y| = m$. Також нехай дано дві дискретні міри
    $\mathbf a \in \mathbb{R}_+^n$ та $\mathbf b \in \mathbb{R}_+^m$ і матриця цін $\mathbf{C} \in \mathbb{R}_+^{n \times m}$.

    Необхідно зайти матрицю $\mathbf{P} \in \mathbf{U}\left(\mathbf a, \mathbf b\right)$ таку, що функціонал
    $$
        \sum_{i, j} \mathbf{P}_{ij} \cdot \mathbf{C}_{ij},
    $$
    досягає мінімуму.
\end{definition}

Означення вище є частковим випадком~\ref{def:kantorovich-problem}, коли обидві міри $\mu$ та $\nu$ є
сумами мір Дірака.

Таким чином для означення вище також справедливі аналоги дуальних теорем~(\ref{theorem:kantorovich-duality}--\ref{theorem:kr-duality}).

\section{Задача пошуку мінімального потоку на графі}

Надалі ми будемо працювати із зваженими графами $G = (V, E, w)$, де $V$ -- скінченна множина вершин графу,
$E \subset V \times V$ -- множина неорієнтованих ребер графу, тобто $(u, v) \in E \iff (v, u) \in E$, 
$w : E \to \mathbb{R}_+ \cup \{+\infty\}$ -- функція ваги ребра;
без втрати загальності вважаємо вершини графу $V$ пронумерованими від $1$ до $n := |V|$.

Визначимо потік на графі.
\begin{definition}[Потік на графі]
    \textbf{Потоком на графі} $G = (V, E, w)$ будемо називати довільне відображення $s : E \to \mathbb{R}_+$.
\end{definition}

Без втрати загальності можемо вважати, що довільна функція $f : V \to \mathbb{R}$ 
є деяким вектором $\mathbf{f} \in \mathbb{R}^{|V|}$, для якого виконується $\forall v \in V: f(v) = \mathbf{f}_v$.

Навідміну від непервоного випадку, в якому $\mathbf{f}$ є вектоним полем, потік на графі має тільки <<інтенсивність>>; напрям руху
визначається ребром, що є аргументом потоку.

\subsection{Диференціальні оператори на графі}
Сформулюємо аналоги диференціальних операторів для графу. Такий підхід дозволить природним шляхом узагальнити поняття
задачі мінімального потоку на випадок графу.

\begin{definition}[Оператор градієнту]
    \label{def:grad}
    Для довільної функції вершини графу $\mathbf f \in \mathbb{R}^{|V|}$ визначимо оператор градієнту
    $\nabla : \mathbb{R}^{|V|} \to \mathbb{R}^{|E|}$ як:
    $$
        \forall (i, j) \in E: (\nabla \mathbf f)_{ij} := \mathbf{f}_i - \mathbf{f}_j.
    $$
\end{definition}

\begin{definition}[Оператор дивергенції]
    \label{def:div}
    Для довільного потоку $\mathbf s \in \mathbb{R}^{|E|}$ визначимо оператор дивергенції
    $\text{div} : \mathbb{R}^{|E|} \to \mathbb{R}^{|V|}$ як:
    $$
        \forall i \in V : \div{s}_i := \sum_{j: (i, j) \in E} \mathbf{s}_{ij} - \mathbf{s}_{ji}.
    $$
\end{definition}

Наступне твердження доводить зв'язок між визначеннями вище.

\begin{claim}[Коректність визначення]
    Нехай дан граф $G = (V, E, w)$ і дві фунції $f : V \to \mathbb{R}$ та $g : E \to \mathbb{R}$.
    Визначимо скалярне множення $\langle f, g \rangle := \sum_{v \in V} f_v \cdot g_v$.

    Тоді справедлива рівність:
    \begin{equation*}
        \langle \nabla f, g \rangle = \langle f, \textnormal{div}(g) \rangle.
    \end{equation*}
\end{claim}
\begin{proof}
    Розкриємо скалярний добуток $\langle \nabla f, g \rangle$:
    \begin{align*}
        &\langle \nabla f, g \rangle = \sum_{e \in E} (\nabla f)_e \cdot g_e =
        \sum_{v \in V} \sum_{u : (v, u) \in E} (\nabla f)_{uv} \cdot g_{uv} = \\
        &=\sum_{v \in V} \sum_{u : (v, u) \in E} (f_u - f_v) \cdot g_{uv} =
        \sum_{v \in V} \sum_{u : (v, u) \in E} f_u \cdot g_{uv} - \\
        &-\sum_{v \in V} \sum_{u : (v, u) \in E} f_v \cdot g_{uv}.
    \end{align*}
    Тепер змінимо порядок підсумовування першої суми. Отримаємо:
    \begin{align*}
        &\sum_{v \in V} \sum_{u : (v, u) \in E} f_v \cdot g_{vu} - \sum_{v \in V} \sum_{u : (v, u) \in E} f_v \cdot g_{uv} = \\
        &= \sum_{v \in V} f_v \sum_{u : (v, u) \in E} (g_{vu} - g_{uv}) =
        \sum_{v \in V} f_v \cdot \text{div}(g) = \langle f, \text{div}(g) \rangle
    \end{align*}
\end{proof}

Таким чином, наведені вище означення мають поведінку, 
притаману неперервним операторам градієнту і дивергенції: оператор дивергенції є спряженим оператор до оператору градієнту
відносно рівномірного розподілу на графі.

\subsection{Потік на графі}
Розглянемо зв'язний граф $G = (V, E, w)$, який моделює деяке місто.
Природньо вважати, що ваги ребер моделюють відстань між районами міста, зокрема $\forall e \in E: w(e) \ge 0$.

На множині вершин визначимо дискретні міри $\mathbf a$ та $\mathbf b$, що, аналогічно до секції~\ref{sec:min-flow-problem}, є
відповідно мірами локального попиту та пропозиції.

Визначимо аналог закону збереження маси~(\ref{eq:equilibrium}): для довільної вершини $v \in V$:
\begin{equation}
    \label{eq:equilibrium-graph}
    \div{\mathbf{s}}_v = \mathbf{a}_v - \mathbf{b}_v.
\end{equation}

Оператор дивергенції~\ref{def:div} визначає кількість <<товару>>, що виходить з вершини.
Таким чином~(\ref{eq:equilibrium-graph}) накладає наступну умову на потік: кількість трафіку з вершини має бути рівною
надлишковому попиту в цій вершині.

Тепер визначимо задачу мінімального потоку на графі.

\begin{definition}[Задача мінімального потоку на графі]
    Нехай дано неорієнтований зв'язний граф з невід'ємними вагами $G = (V, E, w)$. Необхідно знайти такий потік
    $s : E \to \mathbb{R}_+$, що функціонал
    \begin{equation}
        \label{eq:min-flow-graph}
        \sum_{e \in E} \mathbf{s}_e \cdot w_e,
    \end{equation}
    досягає мінімуму і виконується умова збереження маси~(\ref{eq:equilibrium-graph}). 

    Потік, що мініміхує функціонал~(\ref{eq:min-flow-graph}) будемо називати \textbf{мінімальним потоком на графі}.
\end{definition}

Таке формулювання збігає з неперервним аналогом задачі мінімального потоку~\ref{def:min-flow}: ми шукаємо потік, який
мінімізує сумарну інтенсивність. 

\section{Еквівалентність задачі оптимального транспортування і задачі пошуку мінімального потоку для графу}

В цій секції буде доведено основний результат роботи.

\begin{theorem}[Еквівалентність задачі оптимального транспортування і мінімального потоку для графу]
    \label{theorem:equiv}
    Нехай дано зв'язний невід'ємно зважений граф $G = (V, E, w)$ і дві міри $\mathbf a, \mathbf b$.
    Визначимо матрицю цін $\mathbf{C}$ наступним чином
    $$
        \mathbf{C}_{ij} = \text{мінімальна вага шляху між вершинами } i, j
    $$.

    Тоді виконується рівність
    \begin{equation}
        \label{eq:eq-graph}
        \min_{\mathbf{P} \in \left(\mathbf a, \mathbf b\right)} \sum_{i, j} \mathbf{P}_{ij} \cdot \mathbf{C}_{ij} =
        \min \left\{
            \sum_{e \in E} \mathbf{s}_e \cdot w_e : \mathbf{s}_e \text{ задовольняє }\ref{eq:equilibrium-graph}
            \right\}.
    \end{equation}
\end{theorem}
\begin{proof}
    Для доведення рівності покажемо, що для довільного плану транспортування можна отримати потік,
    що задовольняє умові~(\ref{eq:equilibrium-graph}), і з довільного
    адекватного потоку можна отримати план транспортування з однаковими цінами.
    Перше покаже, що оптимальна ціна транспортування не менша, за
    ціну оптимального потоку; друге покаже, що ціна оптимального потоку не менша за ціну оптимального транспортування.

    Позначимо $\mathbf{P} \in U\left(\mathbf{a}, \mathbf{b}\right)$ деякий транспортний план, а
    через $C_{uv}$ -- множина ребер, що сполучають вершини $u, v \in V$ і сума
    $$
    \sum_{e \in C_{uv}} w_e
    $$
    мінімальна. Тобто, $C_{uv}$ -- шлях мінімальної ваги між $u, v \in V$.

    Покладемо $\mathbf{s} = \mathbf{0}$. Наступна процедура перетворить $\mathbf{s}$ в потік,
    що задовольняє~(\ref{eq:equilibrium-graph}):
    \begin{enumerate}
        \item Для кожної пари вершин $(u, v) \in V^2$, якщо значення $\mathbf{P}_{uv}$ більше 0, знайти шлях мінімальної довжини 
        $C_{uv}$
        \item Для кожного ребра $e \in C_{uv}$ збільшити значення $\mathbf{s}_e$ на величину $\mathbf{P}_{e}$
    \end{enumerate}
    Визначимо ціну такого потоку:
    $$
        \sum_{e \in E} \mathbf{s}_e \cdot w_e = \sum_{(u, v) \in E} \mathbf{s}_{uv} \cdot w_{uv}.
    $$
    Для цього розпишемо $\mathbf{s}_{uv}$. Отримаємо:
    $$
        \sum_{(u, v) \in E} w_{uv} \cdot \left(\sum_{(i, j) \in V^2 : (u, v) \in C_{ij}} \mathbf{P}_{ij}\right).
    $$
    Змінимо порядок підсумовування:
    $$
        \sum_{(i, j) \in V^2} \mathbf{P}_{ij} \cdot 
        \left(
            \sum_{(u, v) \in E : (u, v) \in C_{ij}} w_{uv}
        \right).
    $$
    Помітимо, що
    $$
        \mathbf{C}_{ij} = \sum_{(u, v) \in E : (u, v) \in C_{ij}} w_{uv}.
    $$
    за означенням матриці $\mathbf{C}$.
    Таким чином
    $$
        \sum_{e \in E} \mathbf{s}_e \cdot w_e = 
        \sum_{(i, j) \in V^2} \mathbf{P}_{ij} \cdot 
        \left(
            \sum_{(u, v) \in E : (u, v) \in C_{ij}} w_{uv}
        \right) = \sum_{(i, j) \in V^2} \mathbf{P}_{ij} \cdot \mathbf{C}_{ij}.
    $$
    Звідки
    $$
        \min_{\mathbf{P} \in \left(\mathbf a, \mathbf b\right)} \sum_{i, j} \mathbf{P}_{ij} \cdot \mathbf{C}_{ij} \ge
        \min \left\{
            \sum_{e \in E} \mathbf{s}_e \cdot w_e : \mathbf{s}_e \text{ задовольняє }(\ref{eq:equilibrium-graph})
            \right\}.
    $$
    Доведемо зворотню нерівність. Для цього побудуємо транспортний план з потоку.

    Нехай $\mathbf{s}$ - потік, що задовольняє умові~(\ref{eq:equilibrium-graph}). Визначимо орієнтований граф
    $G_s = (V_s, E_s)$, де $V_s = \{v \in V: \exists u \in V: \mathbf{s}_{(u, v)} \ge 0 \lor \mathbf{s}_{(v, u)} \ge 0\}$,
    $E_s = \{e \in E : \mathbf{s}_e \ge 0\}$.
    Тобто граф $G_s$ -- це граф, що складається з усіх вершин і ребер, які застосовуються оптимальним потоком.

    Визначимо множини джерел $S$ та стоків $D$ як
    \begin{eqnarray*}
        S = \{v \in V: \mathbf{a}_v - \mathbf{b}_v < 0\}; \\ 
        D = \{v \in V: \mathbf{a}_v - \mathbf{b}_v > 0\}.
    \end{eqnarray*}

    Нехай $\mathbf{s}$ -- деякий коректний потік. Визначимо нульову матрицю $\mathbf{P}$ розміру $n \times n$.
    \begin{enumerate}
        \item Поки $\mathbf{s} \neq \mathbf{0}$;
        \item Для всіх вершин $s \in S$, для всіх вершин $d \in D$;
        \item Знайти найкоротший шлях $C_{sd}$ графу $G_s$, де $\forall e \in C_{sd} > 0$;
        \item Визначити значення $\delta_{sd} := \min_{e \in C_{sd}} \mathbf{s}_e$;
        \item Додати $\delta$ до $\mathbf{P}_{sd}$;
        \item Для всіх $e \in C_{sd}$ зменшити $\mathbf{s}_e$ на $\delta$.
    \end{enumerate}

    Визначимо ціну такого транспортного плану. Будемо вважати, що під час роботи алгоритму на кроці 3 було виявлено $k$
    шляхів; позначимо $k$-й шлях $C^k$. Таким чином, ціна транспортування рівна
    $$
        \sum_{k} \delta_k  \cdot \sum_{(i, j) \in C^k} w_{ij} = \sum_{k} \sum_{(i, j) \in C^k} (\delta_k \cdot w_{ij}). 
    $$
    Змінимо порядок підсумовування:
    $$
        \sum_{(i, j) \in E_s} \sum_{k: (i, j) \in C^k} (\delta_k \cdot w_{ij}) = \sum_{(i, j) \in E_s} w_{ij}
        \left(
            \sum_{k:(i, j) \in C^k} \delta_k
        \right).
    $$
    Помітимо, що для довільного ребра $(i, j) \in E_s$ вираз 
    \begin{equation}
        \label{eq:some-eq}
        \sum_{k:(i, j) \in C^k} \delta_k,
    \end{equation}
    підраховує $\mathbf{s}_{ij}$. Дійсно, процедура описана вище працює доки потік $\mathbf{s}$ не вичерпається.
    Таким чином, вираз~(\ref{eq:some-eq}) своего роду є <<розшаруванням>> значення $\mathbf{s}_{ij}$.
    Звідки
    \begin{align*}
        &\sum_{(i, j) \in E_s} \sum_{k: (i, j) \in C^k} (\delta_k \cdot w_{ij}) = \sum_{(i, j) \in E_s} w_{ij}
        \left(
            \sum_{k:(i, j) \in C^k} \delta_k
        \right) = \\
        = \sum_{(i, j) \in E_s} w_{ij}\cdot \mathbf{s}_{ij}.
    \end{align*}
    Отже
    $$
        \sum_{i, j} \mathbf{P}_{ij} \cdot \mathbf{C}_{ij} = \sum_{k} \delta_k  \cdot \sum_{(i, j) \in C^k} w_{ij} = 
        \sum_{(i, j) \in E_s} w_{ij} \cdot \mathbf{s}_{ij}.
    $$
    Це доводить нерівність
    $$
        \min_{\mathbf{P} \in \left(\mathbf a, \mathbf b\right)} \sum_{i, j} \mathbf{P}_{ij} \cdot \mathbf{C}_{ij} \le
        \min \left\{
            \sum_{e \in E} \mathbf{s}_e \cdot w_e : \mathbf{s}_e \text{ задовольняє }\ref{eq:equilibrium-graph}
            \right\}.
    $$
    звідки отримуємо твердження теореми.
\end{proof}

Теорему~\ref{theorem:equiv} можна розглянути з іншого боку: нехай кожна вершина $v \in V$ -- це частка з деякою масою
$\mathbf{a}_v - \mathbf{b}_v$, а $e \in E$ -- <<елементарний об'єм>>. В такому разі пошук оптимального плану, це визначення
руху індивідуальної частки $v \in V$ по <<всьому об'єму>>,
а визначення оптимального потоку -- це визначення загальної маси часток, що проходить по конкретному
<<елементарному об'єму>> -- ребру $e \in E$.

Подібні підходи відомі в гідродинаміці (і теорії поля загалом) як \textit{лагранжевий} та \textit{ейлерів} підхід відповідно. 
Обидва підходи є еквівалентними і пов'язані рівнянням непервності, що де-факто є аналогом рівняння збереження маси,
аналогічно до~(\ref{eq:equilibrium-graph}).

\chapconclude{\ref{chap:graph}}
У цьому розділі було надано визначення аналогів диверенціальних операторів, за допомогою яких
було розроблено формулювання задачі Канторовича~\ref{def:kantorovich-problem}
та задачі мінімального потоку~\ref{def:min-flow} на випадок графів.

Було конструктивно доведено еквівалентність задач на графу, шляхом надання відповідних процедур перетворення
транспортного плану на потік і навпаки.

Наступний розділ буде присвячено формулюванню понять, що дозволяють чисельно перевірити коретність, та безпосередньо
перевірці результатів поточного розділу.
%!TEX root = ../thesis.tex
\chapter{Перевірка результатів}
\label{chap:experiment}

У цьому розділі наведено короткі теоретичні відмості про алгоритмічні розв'язки задачі оптимального траснспортування та 
мінімального потоку. Наведено порівняння резлутатів знаходження оптимального транспортування на пряму і використовуючи мінімальний
потік. Наведено аналіз отриманих результатів.

\section{Алгоритмічне підґрунття}
Коротко розглянемо теоретичні відомості, що дозволять перевірити теоретичні результати алгоритмічно.

\subsection{Задача Канторовича}
Наведемо спосіб перетворити дискретні задачу Канторовича~(\ref{def:kantorovich-problem-d}) у задачу лінійного програмування У
у стандартній формі.

\begin{claim}[Задача Канторовича як задача лінійного програмування~\cite{cot}]
    Нехай $\mathbf{C}$ -- матриця цін, $\mathbf{a}, \mathbf{b}$ -- дві дискретні міри
    та $\mathbf{I}_n$ -- одинична матриця $n \times n$. Визначимо матрицю
    $$
    \mathbf{A} = 
    \begin{bmatrix}
    \mathbb{1}_n^\top \otimes \mathbf{I}_m \\
    \mathbf{I}_n \otimes \mathbb{1}_m^\top
    \end{bmatrix}.
    $$
    Визначимо вектор $\mathbf{c}$, як вектор, отриманий шляхом з'єднання колонок матриці $\mathbf{C}$ та вектор $\mathbf{k}$ як
    $$ \mathbf{k} =
        \begin{bmatrix}
            \mathbf{a} \\
            \mathbf{b}
        \end{bmatrix}.
    $$
    Тоді
    \begin{equation}
        \label{eq:kantorovich-linear}
        \min_{\mathbf{P} \in U\left(\mathbf{a}, \mathbf{b}\right)} \sum_{i,j} \mathbf{P}_{ij} \cdot \mathbf{C}_{ij} =
        \min_{\substack{
            \mathbf{p} \in \mathbb{R}_+^{nm} \\
            \mathbf{A}\mathbf{p} = \mathbf{k}
        }}
        \mathbf{c}^\top \mathbf{p}.
    \end{equation}
\end{claim}

Відповідно пошук оптимального плану транспортування можна звести до розв'язання задачі лінійного
програмування~(\ref{eq:kantorovich-linear})

\subsection{Задача мінімального потоку}

Визначимо \textbf{залишковий граф} (\textit{англ.} Residual Network)
\begin{definition}[Залишковий граф]
    Для графа $G = (V, E, w)$ і потоку $\mathbf{s}$ визначимо \textbf{залишковий граф} $G^s = (V, E^s, w^s)$, де
    $E^s = \{e \in E : \mathbf{s}_e > 0\}$ і 
    $$
        w^s_{u, v} = \begin{cases}
            w_{u, v}, &\text{якщо }(u, v) \in E \\
            -w_{v, u}, &\text{якщо }(v, u) \in E
        \end{cases}.
    $$
\end{definition}

Наступне твердження є фундаментом алгоритму усунення циклів
(\textit{англ.} Cycle Cancelling algorithm)~\cite{cycle}.

\begin{theorem}[Критерій мінімальності потоку~\cite{busacker1965finite}]
    Потік $\mathbf{s}$ графу $G$ мінімальний тоді і лише тоді, коли граф $G^s$ немає циклів від'ємної ціни.
\end{theorem}

Отримавши оптимальний потік ми можемо отримати оптимальний план, використавши процедуру,
наведену в доведені теореми~\ref{theorem:equiv}.

\section{Порівняння результатів}
Для порівняння роботи алгоритмів розглянемо декілька графів із <<складною>> системою ребер. Гарним прикладом таких графів
є так звані \textbf{повіні графи}, що мають усі можливі ребра. Ваги ребер, розподіли $\mathbf{a}$ та $\mathbf{b}$
будуть зазначені окремо в таблицях, щоб не засмічувати рисунки. Далі через $K_n$ будемо позначати повний граф на
$n$ вершинах

Спочатку розглянемо повний граф на 5 вершинах $K_5$
\begin{figure}[h]
    \centering
    \completegraph{5}
    \caption{Повний граф на 5 вершинах}
\end{figure}
з відповідними розподілами $\mathbf{a}, \mathbf{b}$ (таблиця~\ref{tab:ex1-ds}) 
\begin{table}[h]
    % Шість колонок три ряди
    \centering
    \begin{tblr}{
        colspec={XXXXXX},
        hlines={-}{},
        vlines={-}{},
        row{1}={c, m, font=\bfseries},
        rows={c, m}
        }
        $v$         &   1&  2&   3&  4&  5 \\
        $\mathbf{a}$   &   6&  4&  10&  8&  9 \\
        $\mathbf{b}$   &   4&  2&  14&  9&  8 \\
    \end{tblr}
    \caption{Таблиця значень розподілів $\mathbf{a}, \mathbf{b}$ на вершинах графу $K_5$}
    \label{tab:ex1-ds}
\end{table}
та вагами
$$
\mathbf{C} = 
\begin{pmatrix}
     0&  7&   6& 10&  5 \\ 
     7&  0&   7&  3&  8 \\
     8&  7&   0&  5&  2 \\
    10&  3&   5&  0&  9 \\
     5&  8&   2&  9&  0
\end{pmatrix}.
$$
Розв'язавши лінійну задачу~(\ref{eq:kantorovich-linear}) ми отримаємо наступний оптимальний план
$$
\mathbf{P}_T^* = 
\begin{pmatrix}
    0& 0& 2& 0& 0 \\
    0& 0& 1& 1& 0 \\
    0& 0& 0& 0& 0 \\
    0& 0& 0& 0& 0 \\
    0& 0& 1& 0& 0 
\end{pmatrix},
$$
з ціною транспортування $24$. 

Порівняємо результат, застосувавши алгоритм усунення циклів і перетворивши оптимальний потік 
на транспортний план. Отримаємо
$$
\mathbf{P}_F^* =
\begin{pmatrix}
    0& 0& 2& 0& 0 \\
    0& 0& 1& 1& 0 \\
    0& 0& 0& 0& 0 \\
    0& 0& 0& 0& 0 \\
    0& 0& 1& 0& 0 
\end{pmatrix},
$$
з ціною $24$.

Таким чином обидва алгоритми досягли мінімального значення рівного $24$.

Проведемо порівняння підходів на більшому графі $K_{7}$. 
\begin{figure}[h]
    \centering
    \completegraph{7}
    \caption{Повний граф на 7 вершинах}
\end{figure}
Розподіли $\mathbf{a}, \mathbf{b}$ подані в таблиці~\ref{tab:ex2-ds}.
Матрицю цін, розподіли $\mathbf{a}, \mathbf{b}$
зробимо менш випадковими. Це дозволить алгоритмам знайти різні оптимальні розв'язки.
$$
\mathbf{C} =
\begin{pmatrix}
    0&2&1&2&1&2&1 \\
    2&0&2&1&2&1&2 \\
    1&2&0&2&1&2&1 \\
    2&1&2&0&2&1&2 \\
    1&2&1&2&0&2&1 \\
    2&1&2&1&2&0&2 \\
    1&2&1&2&1&2&0
\end{pmatrix}.
$$

\begin{table}[h]
    % Вісім колонок три ряди
    \centering
    \begin{tblr}{
        colspec={XXXXXXXX},
        hlines={-}{},
        vlines={-}{},
        row{1}={c, m, font=\bfseries},
        rows={c, m}
        }
        $v$         &    1&   2&   3&   4&   5&   6&  7\\
        $\mathbf{a}$   &    1&   0&   1&   1&   0&   1&  0\\
        $\mathbf{b}$   &    0&   1&   0&   1&   1&   0&  1
    \end{tblr}
    \caption{Таблиця значень розподілів $\mathbf{a}, \mathbf{b}$ на вершинах графу $K_{7}$}
    \label{tab:ex2-ds}
\end{table}

Розв'язання лінійної задачі дає такий оптимальний транспортний план

$$
\mathbf{P}^*_T =
\begin{pmatrix}
    0&  0& 0& 0& 1& 0& 0\\
    0&  0& 0& 0& 0& 0& 0\\
    0&  0& 0& 0& 0& 0& 1\\
    0&  0& 0& 0& 0& 0& 0\\
    0&  0& 0& 0& 0& 0& 0\\
    0&  1& 0& 0& 0& 0& 0\\
    0&  0& 0& 0& 0& 0& 0
\end{pmatrix}.
$$
Ціна транспортування вище рівна $3$. В свою чергу, за допомогою алгоритму усунення циклів було побудовано план

$$
\mathbf{P}^*_F =
\begin{pmatrix}
    0&  0& 0& 0& 0& 0& 1\\
    0&  0& 0& 0& 0& 0& 0\\
    0&  0& 0& 0& 1& 0& 0\\
    0&  0& 0& 0& 0& 0& 0\\
    0&  0& 0& 0& 0& 0& 0\\
    0&  1& 0& 0& 0& 0& 0\\
    0&  0& 0& 0& 0& 0& 0
\end{pmatrix}.
$$
Ціна транспортування також $3$. Хоча план транспортування відрізняється, легко перевірити, що він дійсно переносить
$\mathbf{a}$ на $\mathbf{b}$.

\chapconclude{\ref{chap:experiment}}
В ході цього розділу було розглянуто алгоритмічне підґрунття для розв'язання задач оптимального транспортування і 
мінімального потоку.

Було отримано результати, що свідчать про коректність процедури. 
В першому випадку плани траснспортування співпадають, але в другому ми наочно бачимо різницю. Тим не менш, як було доведено і
перевірино, оптимальна ціна транспортування лишалась сталою.


% Створюємо висновки
\conclusions
%!TEX root = ../thesis.tex
У цій роботі було дослідженно математичну модель задачі побудови оптимального плану транспортування і
задачі побудови мінімального потоку.

У розділі~\ref{chap:review} було розглянуто неперевне формулювання задачі Монжа-Канторовича~(\ref{def:kantorovich-problem})
тоді як параграф~\ref{sec:min-flow-problem} було присвячено задачі пошуку мінімального потоку.

У розділі~\ref{chap:graph} було розгялнуто дискретні формулювання цих задач. В якості природнього дискретного аналогу
метричного простору було обрано зв'язний не орієнтований граф з невід'ємними вагами. Було розроблено аналоги
диференціальних операторів для графу. Наведені аналоги були використані при формулюванні задачі пошуку мінімального потоку.

Також було доведено еквівалентність поставлених задач на графі у теоремі~\ref{theorem:equiv}.

Розділ~\ref{chap:experiment} було присвяченно питання чисельної перевірки одержаних теоретичних результатів. Було
наведено твердження, що дозволило розв'язати задачу Монжа-Канторовича на графі як задачу
лінійного програмування і посилання на алгоритм, для розв'язання задачі мінімального потоку.

Процедура з доведення теореми~\ref{theorem:equiv} була запрограмована мовою Python і наведена в додатку.
Коректність роботи процедури була перевірена на двох прикладах. 
Проведені чисельні експерименти показали, що побудовані двома способами плани є оптимальними, хоча можуть відрізнятись.

% Додаємо бібліографію
% Якщо ви володієте магією bibtex-у, використовуйте її та модифікуйте файл 
% з бібліографією відповідним чином
%\input{Chapters/w2_bibliography}

\printbibliography[title={ПЕРЕЛІК ПОСИЛАНЬ}]

% Створюємо додатки (дивись у файли додатків для необхідних пояснень)
% Якщо ви маєте меншу або більшу кількість додатків, модифікуйте наступні 
% рядки відповідним чином
% Якщо ви не маєте додатків, просто закоментуйте наступні рядки
%!TEX root = ../thesis.tex
\append{Текст програми}
\label{appendix:A}

\section{Програма 1}

Програмний код на мовою Python для побудови оптимального плану транспортування на 
не орієнтованому невід'ємно визначеному графі $G = (V, E, w)$.

\begin{minted}[fontsize=\small]{python}
import numpy as np
import collections

# --- Reusable Edge Class ---
class Edge:
    def __init__(self, to, capacity, cost, reverse_edge_idx):
        self.to = to
        self.capacity = capacity  # Original capacity
        self.cost = cost          # Original cost per unit of flow
        self.reverse_edge_idx = reverse_edge_idx # Index in graph[to] of the reverse edge
        self.flow = 0.0

# --- Edmonds-Karp for Initial Feasible Flow ---
def _bfs_for_edmonds_karp(graph, s, t, num_vertices, parent_nodes, parent_edges):
    """
    Performs BFS to find an augmenting path in the residual graph (capacities only).
    Fills parent_nodes and parent_edges to reconstruct the path.
    Returns True if a path is found, False otherwise.
    """
    for i in range(num_vertices): # Reset parents
        parent_nodes[i] = -1
        parent_edges[i] = None
    
    queue = collections.deque()
    queue.append(s)
    
    visited = [False] * num_vertices
    visited[s] = True

    while queue:
        u = queue.popleft()
        if u == t:
            return True # Path found

        for edge in graph[u]:
            residual_capacity = edge.capacity - edge.flow
            if not visited[edge.to] and residual_capacity > 1e-9: # Check for usable capacity
                visited[edge.to] = True
                parent_nodes[edge.to] = u
                parent_edges[edge.to] = edge
                queue.append(edge.to)
    return False

def _find_initial_feasible_flow_edmonds_karp(graph, s, t, num_vertices, required_flow):
    """
    Finds an initial feasible flow up to required_flow using Edmonds-Karp.
    Modifies flows on the Edge objects in 'graph'.
    Returns the total flow achieved.
    """
    current_total_flow = 0.0
    parent_nodes = [-1] * num_vertices
    parent_edges = [None] * num_vertices # Stores the Edge object leading to a node

    while current_total_flow < required_flow - 1e-9:
        if not _bfs_for_edmonds_karp(graph, s, t, num_vertices, parent_nodes, parent_edges):
            break # No more augmenting paths

        path_bottleneck = float('inf')
        curr = t
        while curr != s:
            edge_to_curr = parent_edges[curr]
            path_bottleneck = min(path_bottleneck, edge_to_curr.capacity - edge_to_curr.flow)
            curr = parent_nodes[curr]
        
        flow_to_send_on_path = min(path_bottleneck, required_flow - current_total_flow)
        if flow_to_send_on_path < 1e-9:
            break # Bottleneck is too small or demand nearly met

        v = t
        while v != s:
            u = parent_nodes[v]
            edge_uv = parent_edges[v] # Edge u -> v
            
            edge_uv.flow += flow_to_send_on_path
            # Update flow on reverse edge v -> u
            graph[v][edge_uv.reverse_edge_idx].flow -= flow_to_send_on_path
            v = u
        
        current_total_flow += flow_to_send_on_path
        
    return current_total_flow

# --- Bellman-Ford for Negative Cycle Detection ---
def _find_negative_cycle_bellman_ford(graph, num_vertices):
    """
    Finds a negative cost cycle in the residual graph using Bellman-Ford.
    The graph contains Edge objects with current flows and original costs.
    Residual capacities and costs are derived from these.
    Returns (list_of_cycle_edges, bottleneck_capacity, cycle_cost) or (None, 0, 0).
    """
    distance = [0.0] * num_vertices # Initialize distances to 0 to find any neg cycle
    predecessor_node = [-1] * num_vertices
    predecessor_edge_obj = [None] * num_vertices # Stores the Edge object from graph

    # Relax edges V-1 times
    for _ in range(num_vertices - 1):
        for u_node in range(num_vertices):
            for edge in graph[u_node]: # edge is an Edge object from u_node to edge.to
                residual_capacity = edge.capacity - edge.flow
                if residual_capacity > 1e-9:
                    # Cost of using this residual edge is edge.cost
                    if distance[edge.to] > distance[u_node] + edge.cost:
                        distance[edge.to] = distance[u_node] + edge.cost
                        predecessor_node[edge.to] = u_node
                        predecessor_edge_obj[edge.to] = edge
    
    # Check for negative cycles (V-th iteration)
    node_on_cycle_candidate = -1
    for u_node in range(num_vertices):
        for edge in graph[u_node]:
            residual_capacity = edge.capacity - edge.flow
            if residual_capacity > 1e-9:
                if distance[edge.to] > distance[u_node] + edge.cost + 1e-9:
                    # Negative cycle detected (or reachable from one)
                    # To ensure 'start_node_for_reconstruction' is actually on the cycle:
                    predecessor_node[edge.to] = u_node # Update for Vth iter relaxation
                    predecessor_edge_obj[edge.to] = edge
                    
                    node_on_cycle_candidate = edge.to 
                    for _ in range(num_vertices):
                        if predecessor_node[node_on_cycle_candidate] == -1: 
                           node_on_cycle_candidate = -1 # Error or no cycle via this path
                           break 
                        node_on_cycle_candidate = predecessor_node[node_on_cycle_candidate]
                    
                    if node_on_cycle_candidate != -1:
                        break # Found a starting point for cycle reconstruction
        if node_on_cycle_candidate != -1:
            break
            
    if node_on_cycle_candidate == -1:
        return None, 0, 0 # No negative cycle found

    # Reconstruct the cycle starting from node_on_cycle_candidate
    cycle_edges = []
    visited_in_reconstruction = [False] * num_vertices
    curr = node_on_cycle_candidate
    
    while not visited_in_reconstruction[curr]:
        visited_in_reconstruction[curr] = True
        # The edge used to reach curr is predecessor_edge_obj[curr]
        # Its source is predecessor_node[curr]
        if predecessor_edge_obj[curr] is None: return None, 0, 0 # Should not happen here
        
        curr = predecessor_node[curr] # Move to the actual start of path that forms cycle.
        
    # Now curr is node_on_cycle_candidate and has been visited.
    # Trace again to collect edges.
    path_to_form_cycle = []
    temp_curr = curr # curr is the start of the cycle (node_on_cycle_candidate after N backtracks)
    
    cycle_cost = 0.0
    cycle_bottleneck = float('inf')

    for _ in range(num_vertices + 1): # Max V edges in a simple cycle
        edge_leading_to_temp_curr = predecessor_edge_obj[temp_curr]
        if edge_leading_to_temp_curr is None: return None, 0, 0 # Should not happen

        path_to_form_cycle.append(edge_leading_to_temp_curr)
        cycle_cost += edge_leading_to_temp_curr.cost
        cycle_bottleneck = min(cycle_bottleneck, 
                               edge_leading_to_temp_curr.capacity - edge_leading_to_temp_curr.flow)
        
        temp_curr = predecessor_node[temp_curr]
        if temp_curr == curr: # We have completed the cycle
            break
    else: # Did not return to start, something is wrong
        return None, 0, 0
        
    path_to_form_cycle.reverse() # To get edges in cycle order

    if cycle_cost < -1e-9 and cycle_bottleneck > 1e-9: # Ensure it's a valid, usable negative cycle
        return path_to_form_cycle, cycle_bottleneck, cycle_cost
    else: # Numerically not a useful negative cycle, or error in reconstruction
        return None, 0, 0


# --- Main Cycle Canceling Algorithm Wrapper ---
def min_cost_cycle_canceling(
    original_graph_edges,
    node_supplies,
    node_demands,
    num_original_nodes
):
    if len(node_supplies) != num_original_nodes:
        raise ValueError(f"Length of node_supplies must equal num_original_nodes.")
    if len(node_demands) != num_original_nodes:
        raise ValueError(f"Length of node_demands must equal num_original_nodes.")

    _supplies_info = []
    _total_supply = 0.0
    for i in range(num_original_nodes):
        if node_supplies[i] < -1e-9: raise ValueError(f"Supply at node {i} cannot be negative.")
        if node_supplies[i] > 1e-9:
            _supplies_info.append((i, node_supplies[i]))
            _total_supply += node_supplies[i]

    _demands_info = []
    _total_demand = 0.0
    for i in range(num_original_nodes):
        if node_demands[i] < -1e-9: raise ValueError(f"Demand at node {i} cannot be negative.")
        if node_demands[i] > 1e-9:
            _demands_info.append((i, node_demands[i]))
            _total_demand += node_demands[i]
    

    super_source_idx = num_original_nodes
    super_sink_idx = num_original_nodes + 1
    num_transformed_nodes = num_original_nodes + 2

    # Build the graph structure for flow algorithms
    # This graph will be modified by initial flow and cycle canceling
    _graph = [[] for _ in range(num_transformed_nodes)]
    # Keep track of original edges to report flow on them
    _original_edge_references = [] 

    def _add_edge_pair_to_graph(u, v, cap, cost, is_original=False):
        fwd_edge = Edge(v, cap, cost, len(_graph[v]))
        bwd_edge = Edge(u, 0, -cost, len(_graph[u])) 
        _graph[u].append(fwd_edge)
        _graph[v].append(bwd_edge)
        if is_original:
            _original_edge_references.append({'u': u, 'v': v, 'edge_obj': fwd_edge, 
                                              'original_capacity': cap, 'original_cost': cost})

    for u, v, cap, cost_val in original_graph_edges:
        if not (0 <= u < num_original_nodes and 0 <= v < num_original_nodes):
            raise ValueError(f"Edge ({u},{v}) has out-of-bounds nodes for original graph.")
        _add_edge_pair_to_graph(u, v, cap, cost_val, is_original=True)

    for s_node, s_amount in _supplies_info:
        _add_edge_pair_to_graph(super_source_idx, s_node, s_amount, 0)
    for d_node, d_amount in _demands_info:
        _add_edge_pair_to_graph(d_node, super_sink_idx, d_amount, 0)
    
    # 1. Find Initial Feasible Flow
    initial_flow_achieved = _find_initial_feasible_flow_edmonds_karp(
        _graph, super_source_idx, super_sink_idx, num_transformed_nodes, _total_supply
    )


    # Calculate initial cost based on the feasible flow
    current_min_cost = 0.0
    for item in _original_edge_references: # Only sum costs for original edges
        current_min_cost += item['edge_obj'].flow * item['original_cost']
    
    # 2. Iteratively Cancel Negative Cycles
    iteration_count = 0
    max_iterations = num_transformed_nodes * num_transformed_nodes * len(original_graph_edges)
    
    while iteration_count < max_iterations : # Add a safety break for complex cases
        iteration_count += 1
        cycle_edge_list, bottleneck_delta, cycle_c = \
        _find_negative_cycle_bellman_ford(_graph, num_transformed_nodes)

        if not cycle_edge_list:
            break # No more negative cycles, current flow is optimal

        # Augment flow along the cycle
        for edge_in_cycle in cycle_edge_list:
            edge_in_cycle.flow += bottleneck_delta
            _graph[edge_in_cycle.to][edge_in_cycle.reverse_edge_idx].flow -= bottleneck_delta
        
        # Update cost: cycle_c is the sum of costs, it's negative
        current_min_cost += bottleneck_delta * cycle_c


    if iteration_count >= max_iterations and cycle_edge_list:
        print("Warning (CC): Reached max iterations, cycle canceling might be slow or stuck.")

    # Prepare final flow distribution for reporting (on original edges)
    final_flow_distribution_original = []
    for item in _original_edge_references:
        edge = item['edge_obj']
        if edge.flow > 1e-9:
            final_flow_distribution_original.append(
                (item['u'], item['v'], edge.flow, item['original_capacity'], item['original_cost'])
            )
            
    return current_min_cost, initial_flow_achieved, final_flow_distribution_original

def flow_decomposition_to_transportation_plan(f, S, D):
    """
    Decomposes a network flow into a Kantorovich transportation plan.
    Handles transshipment nodes.
    """
    # Create a mutable copy of the flow for updates
    residual_flow = collections.defaultdict(lambda: collections.defaultdict(int))
    for u in f:
        for v in f[u]:
            if f[u][v] > 0:
                residual_flow[u][v] = f[u][v]

    transportation_plan = []
    num_nodes = len(S)

    while True:
        source = -1
        # Find a node with a net positive outflow in the residual graph
        for i in range(num_nodes):
            out_flow = sum(residual_flow[i].values())
            in_flow = sum(residual_flow[j][i] for j in range(num_nodes))
            if out_flow > in_flow:
                source = i
                break

        if source == -1:
            break  # No more flow to decompose

        # Find a path from the source to a sink using DFS
        stack = [(source, [source])]
        path_found = None
        
        # Keep track of visited nodes to avoid cycles in the current path search
        visited_in_dfs = {source}

        while stack:
            u, path = stack.pop()
            
            # A sink for a path is a node that is a net demander in the overall problem
            if S[u] - D[u] < 0:
                path_found = path
                break

            for v in sorted(list(residual_flow[u].keys())):
                if residual_flow[u][v] > 0 and v not in visited_in_dfs:
                    visited_in_dfs.add(v)
                    stack.append((v, path + [v]))
        
        if path_found:
            path = path_found
            
            # Determine bottleneck capacity of the found path
            bottleneck = float('inf')
            for i in range(len(path) - 1):
                u_p, v_p = path[i], path[i+1]
                bottleneck = min(bottleneck, residual_flow[u_p][v_p])

            transportation_plan.append((path, bottleneck))

            # Update the residual flow by subtracting the bottleneck amount
            for i in range(len(path) - 1):
                u, v = path[i], path[i+1]
                residual_flow[u][v] -= bottleneck
        else:
            # If no path is found from any source, we are done
            break
            
    return transportation_plan



# --- Example Usage ---
if __name__ == '__main__':
# --- Test Cycle Canceling Algorithm ---
    n = 7

    # Supply and demand arrays
    supplies = np.array([1, 2, 3, 4, 5, 6, 7])
    demands = supplies[::-1]
    
    # Initialize cost matrix
    costMatrix = np.array([
        [0,2,1,2,1,2,1],
        [2,0,2,1,2,1,2],
        [1,2,0,2,1,2,1],
        [2,1,2,0,2,1,2],
        [1,2,1,2,0,2,1],
        [2,1,2,1,2,0,2],
        [1,2,1,2,1,2,0],
    ])


    # Initialize and populate edge list
    edges = []
    for i in range(n):
        for j in range(i + 1, n):
            edges.append(
                (i, j, float("inf"), costMatrix[i, j])
            )
            edges.append(
                (j, i, float("inf"), costMatrix[i, j])
            )

    # Retrieve minimal flow
    cc_cost, cc_flow_moved, cc_flow_details = \
        min_cost_cycle_canceling(
            edges,
            supplies,
            demands,
            n
        )
    
    # Cast flow to a proper format
    flow_dict_of_dicts = collections.defaultdict(lambda: collections.defaultdict(int))
    for u, v, f, _, __ in cc_flow_details:
        flow_dict_of_dicts[u][v] = f
        print((u, v), "->", f)

    # Run the decomposition
    plan = flow_decomposition_to_transportation_plan(flow_dict_of_dicts, supplies, demands)

    # --- Display the Results ---
    print("Decomposition of the Feasible Flow:")
    print("-" * 40)
    total_decomposed_flow = 0
    for path, amount in plan:
        total_decomposed_flow += amount
        print(f"  Path: {' -> '.join(map(str, path))}, Amount: {amount}")
    print("-" * 40)
    print(f"Total flow decomposed: {total_decomposed_flow}")

    # The sum of net supplies should match the total decomposed flow
    total_net_supply = sum(max(0, s_i - d_i) for s_i, d_i in zip(supplies, demands))
    print(f"Total net supply to be shipped: {total_net_supply}")
\end{minted}
% \input{Chapters/z2_appendix_B}


% Нарешті
\end{document}